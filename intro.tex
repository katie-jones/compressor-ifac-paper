Centrifugal gas compressors are employed in a wide range of industrial applications, particularly for gas transportation, extraction and processing.
Compression is an inherently energy-intensive process, with well over 90\% of operating costs spent on energy; small improvements in efficiency can therefore have a significant impact on the operating costs.
At the same time, compressors are critical components in natural gas installations, meaning even short downtimes can also have a large economic impact.
Compressor control systems must therefore maintain the compressor's operating point within its safe operating regime, avoiding instabilities that may damage its machinery and lead to such downtimes.
The most relevant of these instabilities for control is surge, a phenomenon that occurs when the pressure ratio across the compressor is too high for the mass flow, leading to oscillations in mass flow and output pressure, as well as vibrations and an increase in temperature.
Surge can cause permanent damage to machinery in a relatively short time span and it is imperative for industrial compressor control systems to avoid it.

Compressor control thus consists of two sometimes competing goals: process control, wherein a variable (e.g.\ output pressure) is maintained at a reference value, and anti-surge control (ASC), which protects the compressor out against surge conditions.
The performance of a compressor's anti-surge controller is relevant for its efficiency: the most efficient operating points often lie on or near the surge line bounding the unstable surge regime.
Controllers that allow the compressor to be operated closer to surge thus lead directly to an increase in the attainable efficiency (see \cite{Cortinovis2014}).
% An increase in the performance of the ASC thus leads directly to an increase in the attainable efficiency for a given system as the compressor can be operated nearer to the surge line.

The current state-of-the-art in compressor control uses two independent controllers for process control and ASC\@.
The process controller operates by manipulating the speed of the gas turbine or electric driver driving the compressor, while the ASC keeps the compressor away from surge by manipulating the position of a recycle valve.
This valve can be opened to allow flow from the outlet back to the inlet, effectively increasing the mass flow of the compressor and decreasing its pressure ratio, moving the system away from surge.
These controllers are typically implemented using simple PI or PID controllers, with added loop decoupling and hand-tuned open-loop control responses near boundaries to address nonlinearities and control interactions.
Constraints are generally treated using clipping and anti-windup logic, which require further tuning.

Such a decoupled approach is necessary when considering gas turbine-powered compressors, as the dynamics of the turbine are much slower than those that lead to compressor surge.
The transition of compressor control from gas turbines to electric drivers with faster torque responses has, however, opened the door for new, multivariable control algorithms that exploit these faster reaction times.
Such a multivariable controller can take advantage of the quick response of electric driver torque compared to the recycle valve opening to decrease the response time of the ASC to disturbances, thereby increasing its performance.

In recent years, model predictive control (MPC) has been proposed as an alternative to frequency-domain approaches as it can explicitly consider both the coupling and physical constraints that make compressor control so challenging.
\cite{Cortinovis2015} showed that combined process and ASC using MPC led to a reduced distance to surge and a reduced settling time for process control in a centrifugal compressor in experiments compared to the current state-of-the-art.
Similarly, \cite{Budinis2015} found that a combined process/ASC MPC controller better handled disturbances and changes in load pattern compared to a PID controller in simulation.
In \cite{Mercangoz2016} MPC was used to decouple control interactions between process and anti-surge controllers.
\cite{Bentaleb2014} also demonstrated that an MPC controller manipulating driver speed and the inlet guide vanes could achieve better process performance than a PID controller manipulating only the driver speed.
This result was extended in \cite{Bentaleb2015} to include the use of a recycle valve to combine process and anti-surge control.

The major disadvantage of MPC compared to conventional control approaches is its computational complexity, and the resulting difficulty of achieving a sampling rate fast enough to handle the relatively fast dynamics observed in compressors.
Furthermore, in industrial applications, compressors are often combined, either in parallel to increase mass flow, or in series to increase the pressure ratio achieved.
In such systems, as the number of states and inputs increase so does the required computation time of an MPC controller, making traditional MPC impractical in many situations.

In this article a distributed MPC (dMPC) control approach is proposed, which overcomes this limitation by dividing the optimization problem posed by a multi-compressor system into sub-problems to be solved at the individual compressor level.
These sub-problems can be solved in parallel, reducing the required sample time as compared to a centralized MPC solution.
DMPC has received significant attention in the control community and a range of different methods have been proposed for network and communication system configuration (see \cite{Camponogara2002, Venkat2007}).
Adopting the terminology in \cite{Scattolini2009}, these methods can be classified according to the information exchange into non-iterative and iterative algorithms and the type of objective into cooperative and non-cooperative algorithms, as in \cite{Zeilinger2013}.

The use of dMPC implies a loss of performance compared to a traditional, centralized MPC controller.
This loss -- in particular for ASC performance -- are examined for both an iterative cooperative and an iterative non-cooperative dMPC method.
In addition to the controller performance, the computational efficiency of each control approach is evaluated.
Two compressor systems are considered as test cases to evaluate the performance of the dMPC approach in simulation: a parallel and a serial configuration, each with two compressors.
