Two variants of MPC are considered in this work: centralized and distributed control.
Centralized control is considered too computationally expensive to implement in practice, necessitating the development of more efficient distributed controllers, but it is used as a benchmark to evaluate the performance of the distributed controllers.
The system considered is non-linear, however, in order to take advantage of efficient \g{qp} solvers, a linearized approach is taken in this work whereby the non-linear system is re-linearized at each time step and the resulting linear model used to generate the optimization. 
In order to control the relatively fast dynamics leading to compressor surge, the controllers operate at a sampling rate of $\g{ts} = \u{50}{ms}$.
The model used for the MPC controller formulation is the same as the one used in \sect{modelling} to simulate the compressor.
No model mismatch was considered in order to compare the controllers' performance in an ideal scenario.

The nonlinear model is linearized at each sampling instant about the current states (\g{xhat}) and inputs (\gmo{ucurr}), and discretized using the 4th order Runge-Kutta method.
At sampling instant $k$ it is thus given by:

\begin{equation}
  \begin{split}
    \gi{Delta} \gpio{xhat} & = \gi{sys-mats} \gi{Delta}\gpi{xhat} + \gii{sys-mats} \gi{Delta}\gpi{ucurr} + \g{fcurr}\left( \g{xhat}, \gmo{ucurr} \right)\\
    \gi{Delta} \gpi{ycurr} & = \giii{sys-mats} \gi{Delta} \gpi{xhat}, \qquad i\geq 0\\
  \end{split}
\end{equation}

\noindent where \g{fcurr} is the derivative of the system evaluated at the linearization point, \g{xhat} is the state estimate at time instant $k$, \giv{sys-mats}\glsadd{sys-mats} give the model linearized about the current operating point, and \g{Delta} refers to the difference in $\left( \cdot \right)$ relative to the linearization point.

The resulting discrete-time, linear system is then augmented to include both error states as integrators for offset-free control, and delayed input states for the recycle valve.
One integrator for each output is added, as well as 40 delayed states per compressor (multiplied by a sampling rate of \u{50}{ms} to give a total delay of \u{2}{s}).
With these additions, the augmented state of the system (\g{xaug}) is given by:

\begin{equation}
  \g{xaug} =
  \begin{bmatrix}
    \gi{Delta} \g{xhat}\\
    \g{udel}\\
    \g{integrator}
  \end{bmatrix},
  \qquad
  \g{udel} = \tps{\begin{bmatrix} u_{\text{r},k-n\ut{del}+1} & \cdots & u_{\text{r},k-1} & u_{\text{r},k} \end{bmatrix}}
%
  \label{eq:mpc:xaug}
\end{equation}

\noindent where 
\g{xhat} contains the original states of the system, 
\g{udel} contains the delayed recycle valve inputs from the earliest to most recent,
$n\ut{del}$ is the total number of delayed states and
\g{integrator} contains the integrator states.


The augmented system dynamics are then as follows:

\begin{align}
  \begin{split}
    \label{eq:mpc:augmented-state-eqs}
    \gpio{xaug} ={}& 
    \ubrace{\begin{bmatrix}
      \ga{sys-mats} & \begin{bmatrix} \g{Bdelay} & 0\end{bmatrix} & 0 \\[0.5em]
      0 & \begin{bmatrix} 0 & \g{Adelay} \end{bmatrix} & 0\\[0.5em]
      0 & 0 & I_{n\ut{e}\times n\ut{e}}
    \end{bmatrix}}{\ga{augsys-mats}}
    \left( \gpi{xaug} - 
    \begin{bmatrix}
      0\\
      \begin{bmatrix}
        u_{\text{r},k-1}\\
        0\\
      \end{bmatrix}\\
      0
    \end{bmatrix}
    \right)\\
    & + 
    \ubrace{\begin{bmatrix}
      \g{Bnodelay}\\
      \g{Baug}\\
      0
    \end{bmatrix}}{\gb{augsys-mats}}
    \ubrace{\begin{bmatrix}
      u_{\text{r},k+i}\\
      \Delta T_{\text{d},k+i}
    \end{bmatrix}}{\gpi{deltau}}
    + \begin{bmatrix}
      \g{fcurr}\\ 0 \\ 0
    \end{bmatrix}
  \end{split}\\
% y
  \gi{Delta} \g{ycurr} ={}& \ubrace{\begin{bmatrix}
    \gc{sys-mats} & 0 & I_{n\ut{e} \times n\ut{e}}
  \end{bmatrix}}{\gc{augsys-mats}}
  \g{xaug}
  \label{eq:mpc:augmented-output-eqs}
\end{align}


\noindent where
$n\ut{e}$ gives the number of integrator states,
and \giv{augsys-mats}\glsadd{augsys-mats} give the augmented, linearized model of the system.

The first component of \gpi{udel}  is multiplied by \g{Bdelay}, and must therefore be corrected by a factor $u_{\text{r},k-1}$, the value about which the system was linearized. 
The other components of \gpi{udel} are not corrected because they are only multiplied by \g{Adelay} which simply shifts them up a row.

