%===============================================================================
% $Id: ifacconf.tex 19 2011-10-27 09:32:13Z jpuente $  
% Template for IFAC meeting papers
% Copyright (c) 2007-2008 International Federation of Automatic Control
%===============================================================================
\documentclass{ifacconf}

\usepackage{natbib}        % required for bibliography

\input{head/pkgs.tex}
\input{head/defs.tex}
\input{head/glossaries}
\input{head/acronyms}
\input{figures/diagram.tex}

\usetikzlibrary{external}
\tikzexternalize
%===============================================================================
\begin{document}
\begin{frontmatter}

\title{Distributed Model Predictive Control of Centrifugal Compressor Systems} 
% Title, preferably not more than 10 words.

% \thanks[footnoteinfo]{Sponsor and financial support acknowledgment
% goes here. Paper titles should be written in uppercase and lowercase
% letters, not all uppercase.}

\author[First]{K. Jones} 
\author[First]{A. Cortinovis} 
\author[First]{M. Mercangoez}

\address[First]{ABB Switzerland Ltd., Corporate Research, Baden-Dättwil, Switzerland (e-mail: katie.jones@reactive-robotics.com, andrea.cortinovis@ch.abb.com, mehmet.mercangoez@ch.abb.com).}

\begin{abstract}                % Abstract of not more than 250 words.
The performance and computational cost of distributed MPC for the control of compressor networks is investigated in simulation.
Both cooperative and non-cooperative approaches are considered and compared to the performance achieved with centralized control in the presence of a discharge-side disturbance.
Two systems, each with two compressors, are studied: one is arranged in parallel configuration and the other in series.
Due to the high degree of non-linearity of both systems, the models are re-linearized at each time step and a linear, delta MPC formulation is used.
The controller is implemented using the quadratic program solver \qpoases{}, and a \cpp{} implementation is used to evaluate the computational cost of each control approach.
The non-cooperative controller exhibited a significantly reduced computation time relative to the centralized controller (25\% and 40\% lower for the parallel and serial configurations, respectively), while the cooperative controller did not significantly reduce the computation time.
For the parallel configuration, both distributed and centralized controllers had identical control performance.
For the serial configuration, only the cooperative controller achieved similar performance to the centralized approach; the non-cooperative controller demonstrated a 9\% reduction in the minimum surge control distance reached in the downstream compressor.
\end{abstract}

\begin{keyword}
  Distributed mpc, anti-surge control, nonlinear mpc, compressor modeling, variable speed drive
\end{keyword}

\end{frontmatter}
%===============================================================================

\section{Introduction}
Centrifugal gas compressors are employed in a wide range of industrial applications, particularly for gas transportation, extraction and processing.
Compression is an inherently energy-intensive process, with well over 90\% of operating costs spent on energy; small improvements in efficiency therefore have a significant impact on the operating costs.
At the same time, compressors are critical components in natural gas installations, meaning even short downtimes also have a large economic impact.
Compressor control systems must therefore maintain the compressor's operating point within its safe operating regime, avoiding instabilities that may damage its machinery and lead to such downtimes.
The most relevant of these instabilities for control is surge, a phenomenon that occurs when the pressure ratio across the compressor is too high for the mass flow, leading to oscillations in mass flow and output pressure, as well as vibrations and an increase in temperature.
Surge can cause permanent damage to machinery in a relatively short time span and it is imperative for industrial compressor control systems to avoid it.

Compressor control thus consists of two sometimes competing goals: process control, wherein a variable (e.g.\ output pressure) is maintained at a reference value, and anti-surge control (ASC), which protects the compressor against surge conditions.
The performance of a compressor's anti-surge controller is relevant for its efficiency: the most efficient operating points often lie on or near the surge line bounding the unstable surge regime.
Controllers that allow the compressor to be operated closer to surge thus lead directly to an increase in the attainable efficiency (see \cite{Cortinovis2014}).
% An increase in the performance of the ASC thus leads directly to an increase in the attainable efficiency for a given system as the compressor can be operated nearer to the surge line.

The current state-of-the-art in compressor control uses two independent controllers for process control and ASC\@.
The process controller operates by manipulating the speed of the gas turbine or electric driver driving the compressor, while the ASC keeps the compressor away from surge by manipulating the position of a recycle valve.
This valve can be opened to allow flow from the outlet back to the inlet, effectively increasing the mass flow of the compressor and decreasing its pressure ratio, moving the system away from surge.
These controllers are typically implemented using simple PI or PID controllers, with added loop decoupling and hand-tuned open-loop control responses near boundaries to address nonlinearities and control interactions.
Constraints are generally treated using clipping and anti-windup logic, which require further tuning.

Such a decoupled approach is necessary when considering gas turbine-powered compressors, as the dynamics of the turbine are much slower than those that lead to compressor surge.
The transition of compressor control from gas turbines to electric drivers with faster torque responses has, however, opened the door for new, multivariable control algorithms that exploit these faster reaction times.
Such a multivariable controller can take advantage of the quick response of electric driver torque compared to the recycle valve opening to decrease the response time of the ASC to disturbances, thereby increasing its performance.

In recent years, model predictive control (MPC) has been proposed as an alternative to frequency-domain approaches as it can explicitly consider both the coupling and physical constraints that make compressor control so challenging.
\cite{Cortinovis2015} showed that combined process and ASC using MPC led to a reduced distance to surge and a reduced settling time for process control in a centrifugal compressor in experiments compared to the current state-of-the-art.
Similarly, \cite{Budinis2015} found that a combined process/ASC MPC controller better handled disturbances and changes in load pattern compared to a PID controller in simulation.
In \cite{Mercangoz2016} MPC was used to decouple control interactions between process and anti-surge controllers.
\cite{Bentaleb2014} also demonstrated that an MPC controller manipulating driver speed and the inlet guide vanes could achieve better process performance than a PID controller manipulating only the driver speed.
This result was extended in \cite{Bentaleb2015} to include the use of a recycle valve to combine process and anti-surge control.

The major disadvantage of MPC compared to conventional control approaches is its computational complexity, and the resulting difficulty of achieving a sampling rate fast enough to handle the relatively fast dynamics observed in compressors.
Furthermore, in industrial applications, compressors are often combined, either in parallel to increase mass flow, or in series to increase the pressure ratio achieved.
In such systems, as the number of states and inputs increase so does the required computation time of an MPC controller, making traditional MPC impractical.

In this article a distributed MPC (dMPC) control approach is proposed, which overcomes this limitation by dividing the optimization problem posed by a multi-compressor system into sub-problems to be solved at the individual compressor level.
These sub-problems can be solved in parallel, reducing the computation time as compared to a centralized MPC solution.
DMPC has received significant attention in the control community and a range of different methods have been proposed for network and communication system configuration (see \cite{Camponogara2002, Venkat2007}).
Adopting the terminology in \cite{Scattolini2009}, these methods can be classified according to the information exchange into non-iterative and iterative algorithms and the type of objective into cooperative and non-cooperative algorithms, as in \cite{Zeilinger2013}.

The use of dMPC implies a loss of performance compared to a traditional, centralized MPC controller.
This loss -- in particular for ASC performance -- are examined in this article for both an iterative cooperative and an iterative non-cooperative dMPC method.
In addition to the controller performance, the computational efficiency of each control approach is evaluated.
Two compressor systems are considered as test cases to evaluate the performance of the dMPC approach in simulation: a parallel and a serial configuration, each with two compressors.


\section{Modelling \& Simulation}
The controllers presented in this work were evaluated in simulation, using a model based on a two-compressor system at an ABB Research Facility. 
The physical setup is described in \cite{Cortinovis2015} for a single compressor.
Two different configurations were tested: a parallel configuration and a serial configuration, each consisting of two compressors which, for simplicity, were assumed to be identical.

The single compressor unit used in this work refers to a centrifugal compressor driven by a variable-speed electric driver, with a tank at both the inlet (suction) and outlet (discharge).
Both the suction and discharge tanks are connected to atmospheric pressure by a valve.
These suction and discharge valves represent the flow conditions upstream and downstream of the compressor, respectively, such that various flow conditions and disturbances can be simulated by simply changing the position of the valves.
The artificial disturbances applied using these valves are designed to mimic situations frequently encountered in industrial applications (e.g. shutdown or startup of a downstream compressor).
The compressor model used is based on the Gravdahl-Greitzer model described in \cite{Gravdahl1999}. 
It is defined by five states: the suction (\g{ps}) and discharge (\g{pd}) pressures, the mass flow through the compressor (\g{qc}), the rotational speed of the compressor (\g{omegac}), and the mass flow rate through the recycle valve (\g{qr}).
The two inputs to the system are the torque applied to the electric driver (\g{torque}) and the position of the recycle valve (\g{ur}).
The controlled outputs are the output pressure (\g{pout}) and the surge distance (\g{sd}), a measure of how far the compressor is from surge.
The interested reader may refer to \cite{Cortinovis2015} for a more detailed description of the model and its identification.

The parallel and serial compressor systems studied in this work are shown in Figure~\ref{fig:comp-systems-diagrams}.
In the parallel case, the two compressors' outlet tanks both discharge into a single discharge tank and the pressure in this discharge tank (\g{pt}) is the primary process variable.
This pressure also serves as an additional state while the inputs and outputs remain the same as for a single compressor.
For the serial configuration, the upstream compressor discharges directly into the inlet of the downstream compressor.
The states, inputs and outputs are the same as those of the individual compressors.

\begin{figure}
  \begin{subfigure}{\linewidth}
    \togglefalse{compleg}
\togglefalse{complabel}
\resizebox{\linewidth}{!}{%
  \large
  \begin{tikzpicture}
      % draw first compressor
    \renewcommand{\glsgraphcmd}[1]{\glsentryuseri{#1}}
    \drawcomp

      % Draw second compressor shifted
    \begin{scope}[yscale=1,xscale=-1,shift={(-24,0)},nodes={xscale=1}]
      \renewcommand{\glsgraphcmd}[1]{\glsentryuserii{#1}}
      \drawcomp
    \end{scope}

    \renewcommand{\glsgraphcmd}[1]{\glsentryname{#1}}

      % Increase size of bounding box
    % \useasboundingbox ($(current bounding box.south east) - (0,2)$) rectangle (current bounding box.north west |- 0,17);

    % \node[align=center] (origin) at (12.5,3.0) {Common\\Tank\\\statetext{\glsgraph{pt}}};
    \coordinate (origin) at (12.0, 3.4);

    % Draw common tank and valve
    % \drawtank{(ctank-bl)}{(ctank-tr)}
    \node[align=center,rectangle,minimum height=2.5cm,text width=5cm,draw=tankcolor,line width=\tankthick] (ctank) at (origin) {Common Tank\\\statetext{\glsgraph{pt}}};

    \coordinate (ctank-bl) at (ctank.south west);
    \coordinate (ctank-tr) at (ctank.north east);
    \coordinate (ovalve) at ($(ctank.north) + (0,\valvewidth) + (0,1)$);

    \drawvalve[rotate=90, transform shape]{ovalve}{0.4}
    \node[below, rotate=90, transform shape, align=center] at ($(ovalve)+(\valveheight,0)$) {Common\\Tank Valve};

    % Draw flow arrows
    \draw[->,line width=\arrowthick] (ctank.north) -- ($(ovalve)-(0,\valvewidth)-(0,0.5*\valvethick)$);
    \drawflow[->]{($(ovalve)+(0,\valvewidth)+(0,0.5*\valvethick)$)}{+(0,1)}

    % Label valve
    \draw[<-,draw=inputcolor,line width=2pt] ($(ovalve)-(2*\valveheight,0)$) -- ++(-1,0) node[below right]{\inputtext{\glsgraph{ut}}};


    % Label boundary conditions
    \node at ($(ovalve)+(0,2)$) {\bctext{\glsgraph{pa}}};
    \node at ($(ovalve)+(10.5,0)$) {\bctext{\glsgraph{pa}}};
    \node at ($(ovalve)+(-10.5,0)$) {\bctext{\glsgraph{pa}}};

    % Label individual compressors
    \node[align=center] at ($(origin)-(6,1)$) {\Large\bfseries Compressor 1};
    \node[align=center] at ($(origin)+(6,-1)$) {\Large\bfseries Compressor 2};

    % Legend
    % \drawlegend{($(ovalve)+(3.5,0)$)}
    % \drawlegend{(current bounding box.north east)}

  \end{tikzpicture}
}

    \caption{Parallel compressor system}
  \end{subfigure}
  \\
  \begin{subfigure}{\linewidth}
    \togglefalse{compleg}
\togglefalse{complabel}
\newdimen\xcoord
\newdimen\ycoord
\newdimen\xcoordb
\newdimen\ycoordb

  \centering
\resizebox{0.9\linewidth}{!}{%
  \begin{tikzpicture}
    \renewcommand{\glsgraphcmd}[1]{\glsentryuseri{#1}}
      % draw first compressor
    \drawcomp
      % \draw (current bounding box.south west) rectangle (current bounding box.north east);

    \togglefalse{compfirstvalve}

    \pgfgetlastxy{\xcoord}{\ycoord}
    \renewcommand{\glsgraphcmd}[1]{\glsentryuserii{#1}}

      % Draw second compressor shifted
    \begin{scope}[yscale=-1,xscale=1,shift={($(\xcoord,\ycoord) - (0.5*\tankdim,6.65)$)},transform shape,nodes={xscale=1,yscale=-1}]
      \drawcomp
    \end{scope}

    \pgfgetlastxy{\xcoordb}{\ycoordb}

    % \coordinate (origin) at ($0.5*(\xcoord,\ycoord) + 0.5*(\xcoordb,\ycoordb)$);
    \coordinate (origin) at (\xcoord,\ycoord);
    % Label compressors
    \node at ($(origin) - (4,0)$) {\Large \bfseries Compressor 1};
    \node at ($(origin) + (4,0)$) {\Large \bfseries Compressor 2};
    \node at ($(origin) + (6.2,-0.2)$) {\bctext{\glsgraph{pa}}};
    \node at ($(origin) - (6.0,-2.7)$) {\bctext{\glsgraph{pa}}};

    % Legend
    % \drawlegend{($(current bounding box.north west) + (1,-1)$)}

  \end{tikzpicture}
}

    \caption{Serial compressor system}
  \end{subfigure}
  \caption[Diagram of compressor systems.]{Diagram of compressor systems with inputs and states labelled.}
  \label{fig:comp-systems-diagrams}
\end{figure}




\section{Model Predictive Control Formulation}
Two variants of MPC are considered in this work: centralized and distributed control.
Centralized control is considered too computationally expensive to implement in practice, necessitating the development of more efficient distributed controllers, but it is used as a benchmark to evaluate the performance of the distributed controllers.
The system considered is non-linear, however, in order to take advantage of efficient \g{qp} solvers, a linearized approach is taken in this work whereby the non-linear system is re-linearized at each time step and the resulting linear model used to generate the optimization. 
In order to control the relatively fast dynamics leading to compressor surge, the controllers operate at a sampling rate of $\g{ts} = \u{50}{ms}$.
The model used for the MPC controller formulation is the same as the one used in \sect{modelling} to simulate the compressor.
No model mismatch was considered in order to compare the controllers' performance in an ideal scenario.

The nonlinear model is linearized at each sampling instant about the current states (\g{xhat}) and inputs (\gmo{ucurr}), and discretized using the 4th order Runge-Kutta method.
At sampling instant $k$ it is thus given by:

\begin{equation}
  \begin{split}
    \gi{Delta} \gpio{xhat} & = \gi{sys-mats} \gi{Delta}\gpi{xhat} + \gii{sys-mats} \gi{Delta}\gpi{ucurr} + \g{fcurr}\left( \g{xhat}, \gmo{ucurr} \right)\\
    \gi{Delta} \gpi{ycurr} & = \giii{sys-mats} \gi{Delta} \gpi{xhat}, \qquad i\geq 0\\
  \end{split}
\end{equation}

\noindent where \g{fcurr} is the derivative of the system evaluated at the linearization point, \g{xhat} is the state estimate at time instant $k$, \giv{sys-mats}\glsadd{sys-mats} give the model linearized about the current operating point, and \g{Delta} refers to the difference in $\left( \cdot \right)$ relative to the linearization point.

The resulting discrete-time, linear system is then augmented to include both error states as integrators for offset-free control, and delayed input states for the recycle valve.
One integrator for each output is added, as well as 40 delayed states per compressor (multiplied by a sampling rate of \u{50}{ms} to give a total delay of \u{2}{s}).
With these additions, the augmented state of the system (\g{xaug}) is given by:

\begin{equation}
  \g{xaug} =
  \begin{bmatrix}
    \gi{Delta} \g{xhat}\\
    \g{udel}\\
    \g{integrator}
  \end{bmatrix},
  \qquad
  \g{udel} = \tps{\begin{bmatrix} u_{\text{r},k-n\ut{del}+1} & \cdots & u_{\text{r},k-1} & u_{\text{r},k} \end{bmatrix}}
%
  \label{eq:mpc:xaug}
\end{equation}

\noindent where 
\g{xhat} contains the original states of the system, 
\g{udel} contains the delayed recycle valve inputs from the earliest to most recent,
$n\ut{del}$ is the total number of delayed states and
\g{integrator} contains the integrator states.


The augmented system dynamics are then as follows:

\begin{align}
  \begin{split}
    \label{eq:mpc:augmented-state-eqs}
    \gpio{xaug} ={}& 
    \ubrace{\begin{bmatrix}
      \ga{sys-mats} & \begin{bmatrix} \g{Bdelay} & 0\end{bmatrix} & 0 \\[0.5em]
      0 & \begin{bmatrix} 0 & \g{Adelay} \end{bmatrix} & 0\\[0.5em]
      0 & 0 & I_{n\ut{e}\times n\ut{e}}
    \end{bmatrix}}{\ga{augsys-mats}}
    \left( \gpi{xaug} - 
    \begin{bmatrix}
      0\\
      \begin{bmatrix}
        u_{\text{r},k-1}\\
        0\\
      \end{bmatrix}\\
      0
    \end{bmatrix}
    \right)\\
    & + 
    \ubrace{\begin{bmatrix}
      \g{Bnodelay}\\
      \g{Baug}\\
      0
    \end{bmatrix}}{\gb{augsys-mats}}
    \ubrace{\begin{bmatrix}
      u_{\text{r},k+i}\\
      \Delta T_{\text{d},k+i}
    \end{bmatrix}}{\gpi{deltau}}
    + \begin{bmatrix}
      \g{fcurr}\\ 0 \\ 0
    \end{bmatrix}
  \end{split}\\
% y
  \gi{Delta} \g{ycurr} ={}& \ubrace{\begin{bmatrix}
    \gc{sys-mats} & 0 & I_{n\ut{e} \times n\ut{e}}
  \end{bmatrix}}{\gc{augsys-mats}}
  \g{xaug}
  \label{eq:mpc:augmented-output-eqs}
\end{align}


\noindent where
$n\ut{e}$ gives the number of integrator states,
and \giv{augsys-mats}\glsadd{augsys-mats} give the augmented, linearized model of the system.

The first component of \gpi{udel}  is multiplied by \g{Bdelay}, and must therefore be corrected by a factor $u_{\text{r},k-1}$, the value about which the system was linearized. 
The other components of \gpi{udel} are not corrected because they are only multiplied by \g{Adelay} which simply shifts them up a row.



\section{Simulation Results}
In order to test the controller performance of the various control implementations, a single benchmark disturbance case was chosen.
It is designed to mimic a typical disturbance downstream of the system such as the shutdown of another compressor.
The disturbances used were as follows:

\begin{itemize}
  \item \textbf{parallel}: common tank discharge valve (\g{ut}) 70\% -> 40\% open;
  \item \textbf{serial}: downstream compressor discharge valve (\gii{ud}) 39\% -> 29\% open.
\end{itemize}
In the following plots, the disturbances are persistent and applied at \u{50}{s}, after the system has reached steady state.

The distributed controllers in this section were implemented using 3 controller iterations. 
As discussed in \cite{jones2016}, this number was determined sufficient for convergence of the applied inputs.

\subsection{Parallel System Control Performance}
The time response of each controller for the parallel system is shown in \fig{res:parallel-timeresp}.
The responses obtained using each of the three controllers are virtually identical -- in this case, using a distributed control approach has no performance penalty and a potential decrease in computational cost (see below).
The controller weights used to generate these results are summarized in \tab{res:parallel-weights}.

\begin{table}
  \centering
  \footnotesize
  \caption[Controller weights used for the parallel system.]{Controller weights used for the parallel system. Surge distance is normalized by a factor of 1000 to have similar weight ranges.}%
  \label{tab:res:parallel-weights}
  \begin{tabular}{ccccc}
    \toprule
    & Centralized & Coop. & Non-coop. 1 & Non-coop. 2 \\
    \midrule
    \gi{torque}  & 20 & 19 & 22 & 0 \\
    \gii{torque}  & 20  & 19 & 0 & 22 \\
    \gi{ur}  & 200 & 190 & 220 & 0 \\
    \gii{ur}  & 200 & 190 & 0 & 220 \\
    \gi{sd}  & 1 & 1 & 1 & 0 \\
    \gii{sd}  & 1 & 1 & 0 & 1 \\
    % \multirow{2}{*}{Compressor output pressure}& \gi{pd}  & 0 & 0 & 0 & 0 \\
    % & \gii{pd}  &0 & 0 & 0 & 0 \\
    \g{pt}  & 0.5 & 0.42 & 0.6 & 0.6 \\
    \bottomrule
  \end{tabular}
\end{table}



\begin{figure}
  {\centering\small\textbf{Parallel System}\\Both Compressors\\[0.5em]}
  % \begin{subfigure}{0.48\linewidth}
    \resizebox{0.48\linewidth}{!}{%
      \input{figures/parallel_p.tex}
    }
  % \end{subfigure}
  \hfill
  % \begin{subfigure}{0.48\linewidth}
    \resizebox{0.48\linewidth}{!}{%
      \input{figures/parallel_sd.tex}
    }
  % \end{subfigure}
  \\
  % \begin{subfigure}{0.48\linewidth}
    \resizebox{0.48\linewidth}{!}{%
      \input{figures/parallel_td.tex}
    }
  % \end{subfigure}
  \hfill
  % \begin{subfigure}{0.48\linewidth}
    \resizebox{0.48\linewidth}{!}{%
      \input{figures/parallel_ur.tex}
    }
  % \end{subfigure}
  \caption[Time response of parallel system.]{Comparison of time responses of centralized and distributed controllers. Distributed controllers use 3 solver iterations. The disturbance applied is a closing of the common tank discharge valve from 70\% to 40\% at time \u{50}{s}.}
  \label{fig:res:parallel-timeresp}
\end{figure}

\begin{figure}
  \centering
  \resizebox{0.5\linewidth}{!}{%
    \input{figures/parallel_sd_zoom.tex}
  }
  \caption[Zoomed view of surge distance time response of parallel system.]{Zoomed view of surge distance time response given in \fig{res:parallel-timeresp}.}
  \label{fig:res:parallel-sd-zoom}
\end{figure}

The integral squared error (ISE) and integral absolute error (IAE) are shown in \tab{res:performance:ser-ise} for all controllers.

\begin{table}
  \centering
  \caption{Integral squared error (ISE) and integral absolute error (IAE) measures for serial controllers.}
  \begin{tabular}{ccccccc}
    \toprule
    & \multicolumn{2}{c}{Centralized} & \multicolumn{2}{c}{Cooperative} & \multicolumn{2}{c}{Non-cooperative}\\
    & ISE & IAE & ISE & IAE &ISE & IAE \\
    \midrule
    \gi{torque} &   0.0012 &    0.012 &    0.001 &   0.0079 &   0.0015 &    0.019\\
    \gi{ur} &  8.6e-05 &   0.0037 &  5.6e-05 &   0.0024 &  0.00011 &   0.0053\\
    \gi{pd} &  0.00096 &    0.011 &   0.0006 &   0.0067 &  0.00081 &    0.014\\
    \gi{sd} &    0.052 &    0.063 &    0.031 &    0.041 &    0.078 &     0.13\\
    \gii{torque} &   0.0027 &    0.016 &  0.00095 &   0.0073 &   0.0017 &    0.021\\
    \gii{ur} &  1.1e-05 &   0.0011 &  2.7e-05 &   0.0016 &  3.2e-05 &   0.0026\\
    \gii{pd} &  0.00047 &   0.0051 &  0.00058 &   0.0049 &  0.00029 &    0.005\\
    \gii{sd} &    0.056 &    0.032 &    0.064 &    0.035 &      0.1 &    0.064\\
    \bottomrule
  \end{tabular}
  \label{tab:res:performance:ser-ise}
\end{table}



\subsection{Serial System Control Performance}

The time response of the serial system is shown in \fig{res:serial-timeresp}.
A zoomed view of the initial surge distance response is shown in \fig{res:serial-sd-zoom}.
The controller weights used to generate these results are summarized in \tab{res:serial-weights}.

\begin{table}
  \centering
  \caption[Controller weights used for the serial system.]{Controller weights used for the serial system. Surge distance is normalized by a factor of 1000 to have similar weight ranges.}
  \footnotesize
  \begin{tabular}{ccccc}
    \toprule
    & Centralized & Coop. & Non-coop. 1 & Non-coop. 2 \\
    \midrule
    \gi{torque}  & 20 & 41 & 30 & 0 \\
    \gii{torque}  & 20  & 25 & 0 & 30 \\
    \gi{ur}  & 250 & 500 & 500 & 0 \\
    \gii{ur}  & 250 & 500 & 0 & 500 \\
    \gi{pd}  & 0.2 & 0.75 & 1.9 & 0 \\
    \gii{pd}  & 1 & 1.5 & 0 & 1.9 \\
    \gi{sd}  & 1 & 4 & 3 & 0 \\
    \gii{sd}  & 8 & 8 & 0 & 3 \\
    \bottomrule
  \end{tabular}
  \label{tab:res:serial-weights}
\end{table}


\begin{figure}
  {\centering\small\textbf{Serial System}\\Upstream Compressor\\[0.5em]}
  % \begin{subfigure}{0.48\linewidth}
    \resizebox{0.48\linewidth}{!}{%
      \input{figures/serial_p1.tex}
    }
  % \end{subfigure}
  \hfill
  % \begin{subfigure}{0.48\linewidth}
    \resizebox{0.48\linewidth}{!}{%
      \input{figures/serial_sd1.tex}
    }
  % \end{subfigure}
  \\
  % \begin{subfigure}{0.48\linewidth}
    \resizebox{0.48\linewidth}{!}{%
      \input{figures/serial_td1.tex}
    }
  % \end{subfigure}
  \hfill
  % \begin{subfigure}{0.48\linewidth}
    \resizebox{0.48\linewidth}{!}{%
      \input{figures/serial_ur1.tex}
    }
  % \end{subfigure}
  % \caption[Time response of serial system.]{Comparison of time responses of centralized and distributed controllers. Distributed controllers use 3 solver iterations. The disturbance applied is a closing of the common tank discharge valve from 70\% to 40\% at time \u{50}{s}.}
  % \label{fig:res:serial-timeresp}
% \end{figure}

% \begin{figure}
  % \ContinuedFloat
  {\centering\small\textbf{Serial System}\\Downstream Compressor\\[0.5em]}
  % \begin{subfigure}{0.48\linewidth}
    \resizebox{0.48\linewidth}{!}{%
      \input{figures/serial_p2.tex}
    }
  % \end{subfigure}
  \hfill
  % \begin{subfigure}{0.48\linewidth}
    \resizebox{0.48\linewidth}{!}{%
      \input{figures/serial_sd2.tex}
    }
  % \end{subfigure}
  \\
  % \begin{subfigure}{0.48\linewidth}
    \resizebox{0.48\linewidth}{!}{%
      \input{figures/serial_td2.tex}
    }
  % \end{subfigure}
  \hfill
  % \begin{subfigure}{0.48\linewidth}
    \resizebox{0.48\linewidth}{!}{%
      \input{figures/serial_ur2.tex}
    }
  % \end{subfigure}
  \caption[Time response of serial system.]{Comparison of time responses of centralized and distributed controllers. Distributed controllers use 3 solver iterations. The disturbance applied is a closing of the downstream compressor's discharge valve from 39\% to 29\% at time \u{50}{s}.}
  \label{fig:res:serial-timeresp}
\end{figure}

\begin{figure}
  \resizebox{\linewidth}{!}{%
    % This file was created by matlab2tikz.
%
\definecolor{mycolor1}{rgb}{0.00000,0.44700,0.74100}%
\definecolor{mycolor2}{rgb}{0.85000,0.32500,0.09800}%
\definecolor{mycolor3}{rgb}{0.92900,0.69400,0.12500}%
%
\begin{tikzpicture}

\begin{axis}[%
width=5cm,
height=4cm,
at={(0\linewidth,0\linewidth)},
scale only axis,
xmin=50,
xmax=80,
xlabel={Time [s]},
xmajorgrids,
% ymin=-3.5,
% ymax=0.5,
ymin=-6,
ymax=1,
ylabel={Relative Surge Control Distance [\%]},
ymajorgrids,
axis background/.style={fill=white},
title style={font=\bfseries},
title={Downstream Compressor},
axis x line*=bottom,
axis y line*=left
]
\addplot [color=mycolor1,solid,line width=1.5pt,forget plot]
  table[row sep=crcr]{%
50	0.00842000000000098\\
50.05	0.00842000000000098\\
50.1	0.00830999999999982\\
50.15	0.00663000000000125\\
50.2	-0.0012099999999986\\
50.25	-0.0222599999999993\\
50.3	-0.0639099999999999\\
50.35	-0.132719999999999\\
50.4	-0.232239999999999\\
50.45	-0.36088\\
50.5	-0.512049999999999\\
50.55	-0.675109999999999\\
50.6	-0.837149999999999\\
50.65	-0.984749999999999\\
50.7	-1.10552\\
50.75	-1.18987\\
50.8	-1.23236\\
50.85	-1.23239\\
50.9	-1.19405\\
50.95	-1.12528\\
51	-1.03653\\
51.05	-0.93906\\
51.1	-0.843399999999999\\
51.15	-0.758109999999999\\
51.2	-0.68894\\
51.25	-0.638629999999999\\
51.3	-0.607069999999999\\
51.35	-0.591819999999999\\
51.4	-0.58885\\
51.45	-0.593369999999999\\
51.5	-0.600519999999999\\
51.55	-0.606\\
51.6	-0.606489999999999\\
51.65	-0.5999\\
51.7	-0.585389999999999\\
51.75	-0.56328\\
51.8	-0.53478\\
51.85	-0.501729999999999\\
51.9	-0.466229999999999\\
51.95	-0.430339999999999\\
52	-0.395809999999999\\
52.05	-0.363939999999999\\
52.1	-0.3355\\
52.15	-0.310759999999999\\
52.2	-0.289579999999999\\
52.25	-0.271539999999999\\
52.3	-0.25637\\
52.35	-0.24578\\
52.4	-0.24252\\
52.45	-0.248589999999999\\
52.5	-0.264349999999999\\
52.55	-0.28846\\
52.6	-0.318319999999999\\
52.65	-0.350669999999999\\
52.7	-0.382239999999999\\
52.75	-0.410279999999999\\
52.8	-0.433059999999999\\
52.85	-0.45011\\
52.9	-0.46228\\
52.95	-0.4715\\
53	-0.48042\\
53.05	-0.491929999999999\\
53.1	-0.508659999999999\\
53.15	-0.532579999999999\\
53.2	-0.564799999999999\\
53.25	-0.60544\\
53.3	-0.653709999999999\\
53.35	-0.70813\\
53.4	-0.766789999999999\\
53.45	-0.827649999999999\\
53.5	-0.88857\\
53.55	-0.94764\\
53.6	-1.00359\\
53.65	-1.056\\
53.7	-1.10493\\
53.75	-1.15064\\
53.8	-1.19375\\
53.85	-1.23507\\
53.9	-1.27545\\
53.95	-1.31567\\
54	-1.35633\\
54.05	-1.39785\\
54.1	-1.44041\\
54.15	-1.48401\\
54.2	-1.52849\\
54.25	-1.57359\\
54.3	-1.619\\
54.35	-1.66439\\
54.4	-1.70949\\
54.45	-1.75405\\
54.5	-1.7979\\
54.55	-1.8409\\
54.6	-1.88296\\
54.65	-1.92401\\
54.7	-1.964\\
54.75	-2.00288\\
54.8	-2.04058\\
54.85	-2.07711\\
54.9	-2.11245\\
54.95	-2.14673\\
55	-2.18014\\
55.05	-2.21291\\
55.1	-2.24532\\
55.15	-2.27765\\
55.2	-2.31017\\
55.25	-2.34312\\
55.3	-2.37669\\
55.35	-2.411\\
55.4	-2.44607\\
55.45	-2.47036\\
55.5	-2.45483\\
55.55	-2.38603\\
55.6	-2.27549\\
55.65	-2.14955\\
55.7	-2.00669\\
55.75	-1.84971\\
55.8	-1.68513\\
55.85	-1.52085\\
55.9	-1.36436\\
55.95	-1.22161\\
56	-1.09641\\
56.05	-0.990429999999999\\
56.1	-0.903429999999999\\
56.15	-0.833749999999999\\
56.2	-0.77886\\
56.25	-0.735829999999999\\
56.3	-0.70175\\
56.35	-0.674009999999999\\
56.4	-0.650469999999999\\
56.45	-0.629529999999999\\
56.5	-0.61009\\
56.55	-0.59153\\
56.6	-0.573539999999999\\
56.65	-0.556069999999999\\
56.7	-0.53916\\
56.75	-0.522939999999999\\
56.8	-0.507479999999999\\
56.85	-0.492789999999999\\
56.9	-0.478789999999999\\
56.95	-0.46545\\
57	-0.452889999999999\\
57.05	-0.441349999999999\\
57.1	-0.43106\\
57.15	-0.42221\\
57.2	-0.41489\\
57.25	-0.4091\\
57.3	-0.40478\\
57.35	-0.401769999999999\\
57.4	-0.39988\\
57.45	-0.39887\\
57.5	-0.398529999999999\\
57.55	-0.39863\\
57.6	-0.398999999999999\\
57.65	-0.399539999999999\\
57.7	-0.4002\\
57.75	-0.40103\\
57.8	-0.402139999999999\\
57.85	-0.403659999999999\\
57.9	-0.405799999999999\\
57.95	-0.408729999999999\\
58	-0.412649999999999\\
58.05	-0.417699999999999\\
58.1	-0.42399\\
58.15	-0.431589999999999\\
58.2	-0.440499999999999\\
58.25	-0.450679999999999\\
58.3	-0.46205\\
58.35	-0.47451\\
58.4	-0.487939999999999\\
58.45	-0.50219\\
58.5	-0.51716\\
58.55	-0.532719999999999\\
58.6	-0.54878\\
58.65	-0.56528\\
58.7	-0.58215\\
58.75	-0.599349999999999\\
58.8	-0.616859999999999\\
58.85	-0.634659999999999\\
58.9	-0.652729999999999\\
58.95	-0.671049999999999\\
59	-0.689609999999999\\
59.05	-0.70839\\
59.1	-0.727349999999999\\
59.15	-0.74647\\
59.2	-0.7657\\
59.25	-0.785019999999999\\
59.3	-0.80437\\
59.35	-0.823739999999999\\
59.4	-0.843069999999999\\
59.45	-0.862349999999999\\
59.5	-0.88153\\
59.55	-0.9006\\
59.6	-0.919529999999999\\
59.65	-0.93831\\
59.7	-0.956919999999999\\
59.75	-0.97535\\
59.8	-0.99358\\
59.85	-1.01162\\
59.9	-1.02945\\
59.95	-1.04707\\
60	-1.06446\\
60.05	-1.08161\\
60.1	-1.09853\\
60.15	-1.11521\\
60.2	-1.13163\\
60.25	-1.14778\\
60.3	-1.16367\\
60.35	-1.17927\\
60.4	-1.19459\\
60.45	-1.20962\\
60.5	-1.22434\\
60.55	-1.23876\\
60.6	-1.25287\\
60.65	-1.26666\\
60.7	-1.28014\\
60.75	-1.29329\\
60.8	-1.30613\\
60.85	-1.31863\\
60.9	-1.33082\\
60.95	-1.34268\\
61	-1.35421\\
61.05	-1.36542\\
61.1	-1.37631\\
61.15	-1.38687\\
61.2	-1.3971\\
61.25	-1.40701\\
61.3	-1.4166\\
61.35	-1.42586\\
61.4	-1.43479\\
61.45	-1.4434\\
61.5	-1.45168\\
61.55	-1.45964\\
61.6	-1.46727\\
61.65	-1.47458\\
61.7	-1.48156\\
61.75	-1.48822\\
61.8	-1.49455\\
61.85	-1.50057\\
61.9	-1.50626\\
61.95	-1.51162\\
62	-1.51667\\
62.05	-1.5214\\
62.1	-1.52581\\
62.15	-1.52989\\
62.2	-1.53366\\
62.25	-1.53712\\
62.3	-1.54025\\
62.35	-1.54307\\
62.4	-1.54557\\
62.45	-1.54775\\
62.5	-1.54962\\
62.55	-1.55117\\
62.6	-1.5524\\
62.65	-1.55332\\
62.7	-1.55393\\
62.75	-1.55422\\
62.8	-1.55419\\
62.85	-1.55385\\
62.9	-1.5532\\
62.95	-1.55224\\
63	-1.55096\\
63.05	-1.54937\\
63.1	-1.54746\\
63.15	-1.54525\\
63.2	-1.54272\\
63.25	-1.53989\\
63.3	-1.53674\\
63.35	-1.53329\\
63.4	-1.52952\\
63.45	-1.52545\\
63.5	-1.52107\\
63.55	-1.51639\\
63.6	-1.5114\\
63.65	-1.50611\\
63.7	-1.50052\\
63.75	-1.49462\\
63.8	-1.48843\\
63.85	-1.48193\\
63.9	-1.47515\\
63.95	-1.46806\\
64	-1.46069\\
64.05	-1.45303\\
64.1	-1.44508\\
64.15	-1.43684\\
64.2	-1.42833\\
64.25	-1.41953\\
64.3	-1.41046\\
64.35	-1.40112\\
64.4	-1.39151\\
64.45	-1.38163\\
64.5	-1.3715\\
64.55	-1.3611\\
64.6	-1.35046\\
64.65	-1.33957\\
64.7	-1.32843\\
64.75	-1.31706\\
64.8	-1.30545\\
64.85	-1.29362\\
64.9	-1.28156\\
64.95	-1.26929\\
65	-1.25681\\
65.05	-1.24412\\
65.1	-1.23124\\
65.15	-1.21816\\
65.2	-1.2049\\
65.25	-1.19147\\
65.3	-1.17786\\
65.35	-1.16408\\
65.4	-1.15015\\
65.45	-1.13607\\
65.5	-1.12185\\
65.55	-1.10749\\
65.6	-1.09301\\
65.65	-1.07841\\
65.7	-1.0637\\
65.75	-1.04889\\
65.8	-1.03398\\
65.85	-1.01898\\
65.9	-1.00391\\
65.95	-0.988779999999999\\
66	-0.973579999999999\\
66.05	-0.958329999999999\\
66.1	-0.94304\\
66.15	-0.92772\\
66.2	-0.912369999999999\\
66.25	-0.897019999999999\\
66.3	-0.881659999999999\\
66.35	-0.8663\\
66.4	-0.85096\\
66.45	-0.835649999999999\\
66.5	-0.820359999999999\\
66.55	-0.805129999999999\\
66.6	-0.78994\\
66.65	-0.774819999999999\\
66.7	-0.75977\\
66.75	-0.7448\\
66.8	-0.729919999999999\\
66.85	-0.715129999999999\\
66.9	-0.70044\\
66.95	-0.685859999999999\\
67	-0.671399999999999\\
67.05	-0.65706\\
67.1	-0.642849999999999\\
67.15	-0.628769999999999\\
67.2	-0.614819999999999\\
67.25	-0.601019999999999\\
67.3	-0.58735\\
67.35	-0.57384\\
67.4	-0.56047\\
67.45	-0.54726\\
67.5	-0.534199999999999\\
67.55	-0.521299999999999\\
67.6	-0.508559999999999\\
67.65	-0.495979999999999\\
67.7	-0.48356\\
67.75	-0.471299999999999\\
67.8	-0.45921\\
67.85	-0.447279999999999\\
67.9	-0.435519999999999\\
67.95	-0.423929999999999\\
68	-0.4125\\
68.05	-0.401249999999999\\
68.1	-0.39016\\
68.15	-0.379239999999999\\
68.2	-0.36849\\
68.25	-0.357919999999999\\
68.3	-0.34751\\
68.35	-0.337269999999999\\
68.4	-0.327199999999999\\
68.45	-0.317309999999999\\
68.5	-0.30758\\
68.55	-0.298019999999999\\
68.6	-0.288629999999999\\
68.65	-0.2794\\
68.7	-0.27035\\
68.75	-0.261449999999999\\
68.8	-0.25273\\
68.85	-0.24417\\
68.9	-0.23577\\
68.95	-0.22753\\
69	-0.21946\\
69.05	-0.211539999999999\\
69.1	-0.20379\\
69.15	-0.19619\\
69.2	-0.18875\\
69.25	-0.18146\\
69.3	-0.17432\\
69.35	-0.167339999999999\\
69.4	-0.160509999999999\\
69.45	-0.153829999999999\\
69.5	-0.1473\\
69.55	-0.14091\\
69.6	-0.13467\\
69.65	-0.128579999999999\\
69.7	-0.122629999999999\\
69.75	-0.116819999999999\\
69.8	-0.11115\\
69.85	-0.10562\\
69.9	-0.10023\\
69.95	-0.0949799999999996\\
70	-0.0898599999999998\\
70.05	-0.0848699999999987\\
70.1	-0.0800199999999993\\
70.15	-0.0752999999999986\\
70.2	-0.0707100000000001\\
70.25	-0.0662399999999987\\
70.3	-0.0619099999999992\\
70.35	-0.0576899999999991\\
70.4	-0.053609999999999\\
70.45	-0.0496400000000001\\
70.5	-0.0457900000000002\\
70.55	-0.0420699999999989\\
70.6	-0.0384599999999988\\
70.65	-0.0349599999999999\\
70.7	-0.0315799999999999\\
70.75	-0.028319999999999\\
70.8	-0.0251599999999996\\
70.85	-0.0221099999999996\\
70.9	-0.019169999999999\\
70.95	-0.01633\\
71	-0.0136000000000003\\
71.05	-0.0109699999999986\\
71.1	-0.00842999999999883\\
71.15	-0.00600000000000023\\
71.2	-0.00366\\
71.25	-0.00141999999999953\\
71.3	0.000740000000000407\\
71.35	0.00280000000000058\\
71.4	0.00477000000000061\\
71.45	0.00666000000000011\\
71.5	0.00846000000000124\\
71.55	0.0101800000000001\\
71.6	0.0118200000000002\\
71.65	0.0133799999999997\\
71.7	0.0148600000000005\\
71.75	0.0162700000000005\\
71.8	0.0176100000000012\\
71.85	0.0188699999999997\\
71.9	0.0200600000000009\\
71.95	0.0211900000000007\\
72	0.0222500000000014\\
72.05	0.0232400000000013\\
72.1	0.0241699999999998\\
72.15	0.0250500000000002\\
72.2	0.0258599999999998\\
72.25	0.0266200000000012\\
72.3	0.0273200000000013\\
72.35	0.0279699999999998\\
72.4	0.0285600000000006\\
72.45	0.0291100000000011\\
72.5	0.0296099999999999\\
72.55	0.0300600000000006\\
72.6	0.0304599999999997\\
72.65	0.0308200000000003\\
72.7	0.0311400000000006\\
72.75	0.0314200000000007\\
72.8	0.0316600000000005\\
72.85	0.03186\\
72.9	0.0320300000000007\\
72.95	0.0321600000000011\\
73	0.0322600000000008\\
73.05	0.03233\\
73.1	0.0323600000000006\\
73.15	0.0323700000000002\\
73.2	0.032350000000001\\
73.25	0.0323100000000007\\
73.3	0.0322300000000002\\
73.35	0.0321400000000001\\
73.4	0.032020000000001\\
73.45	0.031880000000001\\
73.5	0.0317300000000014\\
73.55	0.0315500000000011\\
73.6	0.0313499999999998\\
73.65	0.0311400000000006\\
73.7	0.0309100000000004\\
73.75	0.0306700000000006\\
73.8	0.0304200000000012\\
73.85	0.0301500000000008\\
73.9	0.0298700000000007\\
73.95	0.029580000000001\\
74	0.02928\\
74.05	0.0289700000000011\\
74.1	0.0286600000000004\\
74.15	0.02834\\
74.2	0.0280100000000001\\
74.25	0.0276700000000005\\
74.3	0.027330000000001\\
74.35	0.0269900000000014\\
74.4	0.0266500000000001\\
74.45	0.0263000000000009\\
74.5	0.0259499999999999\\
74.55	0.0256000000000007\\
74.6	0.0252499999999998\\
74.65	0.0249000000000006\\
74.7	0.0245500000000014\\
74.75	0.0242100000000001\\
74.8	0.0238600000000009\\
74.85	0.0235200000000013\\
74.9	0.02318\\
74.95	0.02285\\
75	0.0225200000000001\\
75.05	0.0221900000000002\\
75.1	0.0218699999999998\\
75.15	0.0215600000000009\\
75.2	0.0212500000000002\\
75.25	0.0209500000000009\\
75.3	0.0206499999999998\\
75.35	0.0203600000000002\\
75.4	0.0200800000000001\\
75.45	0.0198100000000014\\
75.5	0.0195500000000006\\
75.55	0.0192899999999998\\
75.6	0.0190400000000004\\
75.65	0.0188100000000002\\
75.7	0.01858\\
75.75	0.0183600000000013\\
75.8	0.0181500000000003\\
75.85	0.0179500000000008\\
75.9	0.0177600000000009\\
75.95	0.0175800000000006\\
76	0.0174200000000013\\
76.05	0.0172600000000003\\
76.1	0.0171100000000006\\
76.15	0.0169800000000002\\
76.2	0.0168600000000012\\
76.25	0.0167400000000004\\
76.3	0.0166400000000007\\
76.35	0.0165500000000005\\
76.4	0.01647\\
76.45	0.0164100000000005\\
76.5	0.016350000000001\\
76.55	0.0163100000000007\\
76.6	0.0162800000000001\\
76.65	0.0162600000000008\\
76.7	0.0162500000000012\\
76.75	0.0162500000000012\\
76.8	0.0162700000000005\\
76.85	0.0163000000000011\\
76.9	0.0163400000000014\\
76.95	0.0163900000000012\\
77	0.0164500000000007\\
77.05	0.0165300000000013\\
77.1	0.01661\\
77.15	0.0167099999999998\\
77.2	0.0168200000000009\\
77.25	0.01694\\
77.3	0.01708\\
77.35	0.01722\\
77.4	0.0173800000000011\\
77.45	0.0175400000000003\\
77.5	0.0177200000000006\\
77.55	0.0179100000000005\\
77.6	0.0181100000000001\\
77.65	0.018320000000001\\
77.7	0.0185500000000012\\
77.75	0.0187800000000014\\
77.8	0.0190300000000008\\
77.85	0.0192800000000002\\
77.9	0.0195500000000006\\
77.95	0.0198200000000011\\
78	0.0201100000000007\\
78.05	0.0204000000000004\\
78.1	0.0207100000000011\\
78.15	0.0210300000000014\\
78.2	0.02135\\
78.25	0.0216900000000013\\
78.3	0.0220300000000009\\
78.35	0.0223899999999997\\
78.4	0.0227500000000003\\
78.45	0.0231300000000001\\
78.5	0.0235099999999999\\
78.55	0.0239000000000011\\
78.6	0.0243000000000002\\
78.65	0.0247100000000007\\
78.7	0.0251200000000011\\
78.75	0.0255500000000008\\
78.8	0.0259800000000006\\
78.85	0.0264300000000013\\
78.9	0.0268800000000002\\
78.95	0.0273400000000006\\
79	0.0278000000000009\\
79.05	0.0282700000000009\\
79.1	0.0287600000000001\\
79.15	0.0292399999999997\\
79.2	0.0297400000000003\\
79.25	0.0302400000000009\\
79.3	0.0307500000000012\\
79.35	0.031270000000001\\
79.4	0.0317900000000009\\
79.45	0.0323200000000003\\
79.5	0.0328600000000012\\
79.55	0.0334000000000003\\
79.6	0.0339500000000008\\
79.65	0.0345000000000013\\
79.7	0.035070000000001\\
79.75	0.0356300000000012\\
79.8	0.0362000000000009\\
79.85	0.0367800000000003\\
79.9	0.037370000000001\\
79.95	0.03796\\
80	0.0385500000000008\\
};
\addplot [color=mycolor2,dashed,line width=1.5pt,forget plot]
  table[row sep=crcr]{%
50	0.00472000000000072\\
50.05	0.00472000000000072\\
50.1	0.00461000000000134\\
50.15	0.00285000000000046\\
50.2	-0.00546999999999898\\
50.25	-0.0279699999999998\\
50.3	-0.0728899999999992\\
50.35	-0.14678\\
50.4	-0.252549999999999\\
50.45	-0.389119999999999\\
50.5	-0.550759999999999\\
50.55	-0.727139999999999\\
50.6	-0.905029999999999\\
50.65	-1.07028\\
50.7	-1.20986\\
50.75	-1.31357\\
50.8	-1.3755\\
50.85	-1.3946\\
50.9	-1.37461\\
50.95	-1.32314\\
51	-1.25039\\
51.05	-1.16746\\
51.1	-1.0848\\
51.15	-1.01101\\
51.2	-0.951989999999999\\
51.25	-0.910699999999999\\
51.3	-0.887269999999999\\
51.35	-0.879549999999999\\
51.4	-0.88374\\
51.45	-0.89521\\
51.5	-0.909179999999999\\
51.55	-0.921379999999999\\
51.6	-0.928439999999999\\
51.65	-0.928179999999999\\
51.7	-0.919689999999999\\
51.75	-0.903189999999999\\
51.8	-0.879849999999999\\
51.85	-0.85145\\
51.9	-0.820069999999999\\
51.95	-0.78773\\
52	-0.756219999999999\\
52.05	-0.72681\\
52.1	-0.70027\\
52.15	-0.67681\\
52.2	-0.656179999999999\\
52.25	-0.637759999999999\\
52.3	-0.62076\\
52.35	-0.60429\\
52.4	-0.587899999999999\\
52.45	-0.573429999999999\\
52.5	-0.56406\\
52.55	-0.562259999999999\\
52.6	-0.568779999999999\\
52.65	-0.582479999999999\\
52.7	-0.600719999999999\\
52.75	-0.620119999999999\\
52.8	-0.63728\\
52.85	-0.649539999999999\\
52.9	-0.655559999999999\\
52.95	-0.655539999999999\\
53	-0.65107\\
53.05	-0.64479\\
53.1	-0.63986\\
53.15	-0.63932\\
53.2	-0.64565\\
53.25	-0.660379999999999\\
53.3	-0.683949999999999\\
53.35	-0.71573\\
53.4	-0.754219999999999\\
53.45	-0.79735\\
53.5	-0.8428\\
53.55	-0.888349999999999\\
53.6	-0.932119999999999\\
53.65	-0.97279\\
53.7	-1.00967\\
53.75	-1.04267\\
53.8	-1.07222\\
53.85	-1.09913\\
53.9	-1.12437\\
53.95	-1.14894\\
54	-1.17352\\
54.05	-1.19871\\
54.1	-1.22497\\
54.15	-1.25248\\
54.2	-1.28142\\
54.25	-1.31207\\
54.3	-1.3445\\
54.35	-1.37837\\
54.4	-1.41324\\
54.45	-1.44864\\
54.5	-1.48409\\
54.55	-1.51919\\
54.6	-1.55363\\
54.65	-1.5872\\
54.7	-1.61981\\
54.75	-1.65148\\
54.8	-1.68225\\
54.85	-1.71224\\
54.9	-1.74156\\
54.95	-1.77035\\
55	-1.79871\\
55.05	-1.82675\\
55.1	-1.85457\\
55.15	-1.88224\\
55.2	-1.90984\\
55.25	-1.93748\\
55.3	-1.96534\\
55.35	-1.99354\\
55.4	-2.02216\\
55.45	-2.05124\\
55.5	-2.08074\\
55.55	-2.11058\\
55.6	-2.14064\\
55.65	-2.17081\\
55.7	-2.20093\\
55.75	-2.2309\\
55.8	-2.2606\\
55.85	-2.28998\\
55.9	-2.31897\\
55.95	-2.33591\\
56	-2.32314\\
56.05	-2.28461\\
56.1	-2.20445\\
56.15	-2.07872\\
56.2	-1.92513\\
56.25	-1.77291\\
56.3	-1.62001\\
56.35	-1.4667\\
56.4	-1.31595\\
56.45	-1.17168\\
56.5	-1.03758\\
56.55	-0.916429999999999\\
56.6	-0.80976\\
56.65	-0.717829999999999\\
56.7	-0.639869999999999\\
56.75	-0.574339999999999\\
56.8	-0.519259999999999\\
56.85	-0.472529999999999\\
56.9	-0.43209\\
56.95	-0.39615\\
57	-0.363219999999999\\
57.05	-0.332129999999999\\
57.1	-0.30204\\
57.15	-0.272329999999999\\
57.2	-0.24257\\
57.25	-0.212569999999999\\
57.3	-0.182639999999999\\
57.35	-0.153309999999999\\
57.4	-0.1252\\
57.45	-0.0988399999999992\\
57.5	-0.0746000000000002\\
57.55	-0.0526799999999987\\
57.6	-0.0330899999999996\\
57.65	-0.0156700000000001\\
57.7	-0.000169999999998893\\
57.75	0.0137700000000009\\
57.8	0.0265000000000004\\
57.85	0.0383800000000001\\
57.9	0.0497300000000003\\
57.95	0.060830000000001\\
58	0.0718800000000002\\
58.05	0.0830300000000008\\
58.1	0.0943500000000004\\
58.15	0.105840000000001\\
58.2	0.117420000000001\\
58.25	0.12893\\
58.3	0.14015\\
58.35	0.150830000000001\\
58.4	0.160690000000001\\
58.45	0.16947\\
58.5	0.17694\\
58.55	0.18291\\
58.6	0.187230000000001\\
58.65	0.189810000000001\\
58.7	0.190610000000001\\
58.75	0.18965\\
58.8	0.186970000000001\\
58.85	0.18267\\
58.9	0.17685\\
58.95	0.169640000000001\\
59	0.161160000000001\\
59.05	0.151540000000001\\
59.1	0.140890000000001\\
59.15	0.12932\\
59.2	0.11692\\
59.25	0.103770000000001\\
59.3	0.0899400000000004\\
59.35	0.0754999999999999\\
59.4	0.0604899999999997\\
59.45	0.0449700000000011\\
59.5	0.0289800000000007\\
59.55	0.0125700000000002\\
59.6	-0.00419999999999909\\
59.65	-0.0213000000000001\\
59.7	-0.0386600000000001\\
59.75	-0.056239999999999\\
59.8	-0.0739999999999998\\
59.85	-0.0918700000000001\\
59.9	-0.109829999999999\\
59.95	-0.12782\\
60	-0.14582\\
60.05	-0.16379\\
60.1	-0.18169\\
60.15	-0.19952\\
60.2	-0.217239999999999\\
60.25	-0.23485\\
60.3	-0.25233\\
60.35	-0.26967\\
60.4	-0.28687\\
60.45	-0.3039\\
60.5	-0.320779999999999\\
60.55	-0.337479999999999\\
60.6	-0.353999999999999\\
60.65	-0.37034\\
60.7	-0.386469999999999\\
60.75	-0.402399999999999\\
60.8	-0.418099999999999\\
60.85	-0.43357\\
60.9	-0.44879\\
60.95	-0.46376\\
61	-0.47845\\
61.05	-0.492859999999999\\
61.1	-0.50698\\
61.15	-0.520799999999999\\
61.2	-0.53431\\
61.25	-0.547499999999999\\
61.3	-0.56037\\
61.35	-0.572909999999999\\
61.4	-0.585129999999999\\
61.45	-0.59702\\
61.5	-0.608569999999999\\
61.55	-0.619789999999999\\
61.6	-0.630679999999999\\
61.65	-0.64124\\
61.7	-0.651459999999999\\
61.75	-0.66135\\
61.8	-0.670909999999999\\
61.85	-0.680129999999999\\
61.9	-0.68903\\
61.95	-0.697589999999999\\
62	-0.70583\\
62.05	-0.713729999999999\\
62.1	-0.721309999999999\\
62.15	-0.72856\\
62.2	-0.735479999999999\\
62.25	-0.742089999999999\\
62.3	-0.74837\\
62.35	-0.754339999999999\\
62.4	-0.759989999999999\\
62.45	-0.765319999999999\\
62.5	-0.77035\\
62.55	-0.775069999999999\\
62.6	-0.779489999999999\\
62.65	-0.7836\\
62.7	-0.787419999999999\\
62.75	-0.79095\\
62.8	-0.79418\\
62.85	-0.797129999999999\\
62.9	-0.79979\\
62.95	-0.802169999999999\\
63	-0.80427\\
63.05	-0.8061\\
63.1	-0.807659999999999\\
63.15	-0.808949999999999\\
63.2	-0.80997\\
63.25	-0.810739999999999\\
63.3	-0.811249999999999\\
63.35	-0.8115\\
63.4	-0.811509999999999\\
63.45	-0.811279999999999\\
63.5	-0.8108\\
63.55	-0.81009\\
63.6	-0.80915\\
63.65	-0.80798\\
63.7	-0.8066\\
63.75	-0.804989999999999\\
63.8	-0.80317\\
63.85	-0.80115\\
63.9	-0.79892\\
63.95	-0.796499999999999\\
64	-0.79388\\
64.05	-0.791079999999999\\
64.1	-0.78809\\
64.15	-0.78492\\
64.2	-0.781579999999999\\
64.25	-0.77807\\
64.3	-0.774399999999999\\
64.35	-0.770569999999999\\
64.4	-0.766579999999999\\
64.45	-0.762449999999999\\
64.5	-0.75817\\
64.55	-0.753749999999999\\
64.6	-0.74919\\
64.65	-0.744509999999999\\
64.7	-0.739699999999999\\
64.75	-0.734769999999999\\
64.8	-0.729719999999999\\
64.85	-0.724559999999999\\
64.9	-0.719289999999999\\
64.95	-0.713909999999999\\
65	-0.70844\\
65.05	-0.70287\\
65.1	-0.697209999999999\\
65.15	-0.691459999999999\\
65.2	-0.68563\\
65.25	-0.67972\\
65.3	-0.673729999999999\\
65.35	-0.667669999999999\\
65.4	-0.661549999999999\\
65.45	-0.655349999999999\\
65.5	-0.6491\\
65.55	-0.64279\\
65.6	-0.63643\\
65.65	-0.63001\\
65.7	-0.62355\\
65.75	-0.617039999999999\\
65.8	-0.61049\\
65.85	-0.603899999999999\\
65.9	-0.59728\\
65.95	-0.590619999999999\\
66	-0.58393\\
66.05	-0.57722\\
66.1	-0.57048\\
66.15	-0.563719999999999\\
66.2	-0.556939999999999\\
66.25	-0.55014\\
66.3	-0.543329999999999\\
66.35	-0.536499999999999\\
66.4	-0.529669999999999\\
66.45	-0.522829999999999\\
66.5	-0.51598\\
66.55	-0.50913\\
66.6	-0.502269999999999\\
66.65	-0.495419999999999\\
66.7	-0.48856\\
66.75	-0.481719999999999\\
66.8	-0.474869999999999\\
66.85	-0.468039999999999\\
66.9	-0.461209999999999\\
66.95	-0.4544\\
67	-0.447589999999999\\
67.05	-0.440799999999999\\
67.1	-0.434029999999999\\
67.15	-0.427269999999999\\
67.2	-0.420529999999999\\
67.25	-0.41381\\
67.3	-0.4071\\
67.35	-0.40042\\
67.4	-0.393769999999999\\
67.45	-0.387129999999999\\
67.5	-0.38052\\
67.55	-0.373939999999999\\
67.6	-0.367389999999999\\
67.65	-0.36086\\
67.7	-0.35436\\
67.75	-0.34789\\
67.8	-0.341449999999999\\
67.85	-0.335039999999999\\
67.9	-0.32867\\
67.95	-0.322319999999999\\
68	-0.316009999999999\\
68.05	-0.30974\\
68.1	-0.3035\\
68.15	-0.297289999999999\\
68.2	-0.291119999999999\\
68.25	-0.28499\\
68.3	-0.278899999999999\\
68.35	-0.27284\\
68.4	-0.266819999999999\\
68.45	-0.260839999999999\\
68.5	-0.254899999999999\\
68.55	-0.248989999999999\\
68.6	-0.24313\\
68.65	-0.237309999999999\\
68.7	-0.23152\\
68.75	-0.225779999999999\\
68.8	-0.220079999999999\\
68.85	-0.21442\\
68.9	-0.208799999999999\\
68.95	-0.203219999999999\\
69	-0.19769\\
69.05	-0.192189999999999\\
69.1	-0.186739999999999\\
69.15	-0.181329999999999\\
69.2	-0.175959999999999\\
69.25	-0.17064\\
69.3	-0.165349999999999\\
69.35	-0.16011\\
69.4	-0.15492\\
69.45	-0.14976\\
69.5	-0.14465\\
69.55	-0.13958\\
69.6	-0.134549999999999\\
69.65	-0.129569999999999\\
69.7	-0.12462\\
69.75	-0.119719999999999\\
69.8	-0.11487\\
69.85	-0.110049999999999\\
69.9	-0.105279999999999\\
69.95	-0.10055\\
70	-0.0958600000000001\\
70.05	-0.0912199999999999\\
70.1	-0.0866100000000003\\
70.15	-0.0820499999999988\\
70.2	-0.0775299999999994\\
70.25	-0.0730500000000003\\
70.3	-0.0686199999999992\\
70.35	-0.0642199999999988\\
70.4	-0.0598599999999987\\
70.45	-0.0555500000000002\\
70.5	-0.0512800000000002\\
70.55	-0.0470399999999991\\
70.6	-0.0428499999999996\\
70.65	-0.0386999999999986\\
70.7	-0.0345899999999997\\
70.75	-0.0305099999999996\\
70.8	-0.0264799999999994\\
70.85	-0.0224899999999995\\
70.9	-0.0185300000000002\\
70.95	-0.014619999999999\\
71	-0.0107400000000002\\
71.05	-0.00689999999999991\\
71.1	-0.00309999999999988\\
71.15	0.000659999999999883\\
71.2	0.00439000000000078\\
71.25	0.00808000000000142\\
71.3	0.01173\\
71.35	0.0153400000000001\\
71.4	0.0189200000000014\\
71.45	0.022450000000001\\
71.5	0.0259600000000013\\
71.55	0.0294300000000014\\
71.6	0.0328600000000012\\
71.65	0.0362500000000008\\
71.7	0.0396099999999997\\
71.75	0.0429399999999998\\
71.8	0.0462300000000013\\
71.85	0.0494900000000005\\
71.9	0.0527100000000011\\
71.95	0.0559000000000012\\
72	0.0590500000000009\\
72.05	0.0621799999999997\\
72.1	0.0652600000000003\\
72.15	0.0683199999999999\\
72.2	0.0713400000000011\\
72.25	0.0743299999999998\\
72.3	0.0772900000000014\\
72.35	0.080210000000001\\
72.4	0.0831100000000013\\
72.45	0.0859700000000014\\
72.5	0.0888000000000009\\
72.55	0.0915999999999997\\
72.6	0.0943700000000014\\
72.65	0.0971100000000007\\
72.7	0.0998200000000011\\
72.75	0.102500000000001\\
72.8	0.10515\\
72.85	0.10777\\
72.9	0.11036\\
72.95	0.112920000000001\\
73	0.115450000000001\\
73.05	0.11796\\
73.1	0.120430000000001\\
73.15	0.12288\\
73.2	0.125300000000001\\
73.25	0.127690000000001\\
73.3	0.130050000000001\\
73.35	0.132390000000001\\
73.4	0.1347\\
73.45	0.136980000000001\\
73.5	0.139240000000001\\
73.55	0.14147\\
73.6	0.14367\\
73.65	0.145850000000001\\
73.7	0.148\\
73.75	0.150130000000001\\
73.8	0.152230000000001\\
73.85	0.154310000000001\\
73.9	0.156360000000001\\
73.95	0.158390000000001\\
74	0.160390000000001\\
74.05	0.162360000000001\\
74.1	0.16432\\
74.15	0.16625\\
74.2	0.168150000000001\\
74.25	0.170030000000001\\
74.3	0.171890000000001\\
74.35	0.173720000000001\\
74.4	0.17553\\
74.45	0.17732\\
74.5	0.17909\\
74.55	0.18083\\
74.6	0.182550000000001\\
74.65	0.18425\\
74.7	0.185920000000001\\
74.75	0.187580000000001\\
74.8	0.189210000000001\\
74.85	0.19082\\
74.9	0.192400000000001\\
74.95	0.19397\\
75	0.19552\\
75.05	0.197040000000001\\
75.1	0.198540000000001\\
75.15	0.20002\\
75.2	0.20149\\
75.25	0.20293\\
75.3	0.20435\\
75.35	0.20575\\
75.4	0.207130000000001\\
75.45	0.208490000000001\\
75.5	0.20983\\
75.55	0.21115\\
75.6	0.21245\\
75.65	0.21373\\
75.7	0.21499\\
75.75	0.216230000000001\\
75.8	0.217450000000001\\
75.85	0.21865\\
75.9	0.219840000000001\\
75.95	0.221\\
76	0.222150000000001\\
76.05	0.223280000000001\\
76.1	0.224390000000001\\
76.15	0.225480000000001\\
76.2	0.226550000000001\\
76.25	0.22761\\
76.3	0.22864\\
76.35	0.229660000000001\\
76.4	0.23066\\
76.45	0.23165\\
76.5	0.232610000000001\\
76.55	0.233560000000001\\
76.6	0.234490000000001\\
76.65	0.2354\\
76.7	0.2363\\
76.75	0.23718\\
76.8	0.23804\\
76.85	0.23888\\
76.9	0.239710000000001\\
76.95	0.24052\\
77	0.24131\\
77.05	0.242090000000001\\
77.1	0.242850000000001\\
77.15	0.243600000000001\\
77.2	0.24433\\
77.25	0.245040000000001\\
77.3	0.24573\\
77.35	0.246410000000001\\
77.4	0.24708\\
77.45	0.247720000000001\\
77.5	0.24836\\
77.55	0.24897\\
77.6	0.24957\\
77.65	0.250160000000001\\
77.7	0.250730000000001\\
77.75	0.251280000000001\\
77.8	0.25182\\
77.85	0.25235\\
77.9	0.25286\\
77.95	0.253350000000001\\
78	0.253830000000001\\
78.05	0.254290000000001\\
78.1	0.25474\\
78.15	0.255180000000001\\
78.2	0.255600000000001\\
78.25	0.25601\\
78.3	0.256400000000001\\
78.35	0.256770000000001\\
78.4	0.25714\\
78.45	0.257490000000001\\
78.5	0.257820000000001\\
78.55	0.258140000000001\\
78.6	0.25845\\
78.65	0.258740000000001\\
78.7	0.259020000000001\\
78.75	0.25929\\
78.8	0.259540000000001\\
78.85	0.259780000000001\\
78.9	0.260010000000001\\
78.95	0.26022\\
79	0.26042\\
79.05	0.26061\\
79.1	0.26078\\
79.15	0.26094\\
79.2	0.261090000000001\\
79.25	0.261230000000001\\
79.3	0.26135\\
79.35	0.261460000000001\\
79.4	0.261560000000001\\
79.45	0.26164\\
79.5	0.261710000000001\\
79.55	0.26177\\
79.6	0.26182\\
79.65	0.26186\\
79.7	0.261890000000001\\
79.75	0.261900000000001\\
79.8	0.261900000000001\\
79.85	0.261890000000001\\
79.9	0.26187\\
79.95	0.26183\\
80	0.261790000000001\\
};
\addplot [color=mycolor3,dotted,line width=1.5pt,forget plot]
  table[row sep=crcr]{%
50	0.0317699999999999\\
50.05	0.0317699999999999\\
50.1	0.0316700000000001\\
50.15	0.0297000000000001\\
50.2	0.0204200000000014\\
50.25	-0.00467999999999869\\
50.3	-0.054829999999999\\
50.35	-0.13749\\
50.4	-0.256489999999999\\
50.45	-0.411049999999999\\
50.5	-0.59571\\
50.55	-0.800979999999999\\
50.6	-1.01456\\
50.65	-1.22283\\
50.7	-1.41256\\
50.75	-1.5725\\
50.8	-1.69469\\
50.85	-1.77544\\
50.9	-1.81552\\
50.95	-1.81994\\
51	-1.79693\\
51.05	-1.75668\\
51.1	-1.70989\\
51.15	-1.66635\\
51.2	-1.63382\\
51.25	-1.61741\\
51.3	-1.61935\\
51.35	-1.6392\\
51.4	-1.67437\\
51.45	-1.72075\\
51.5	-1.77353\\
51.55	-1.82785\\
51.6	-1.87941\\
51.65	-1.92492\\
51.7	-1.96235\\
51.75	-1.99096\\
51.8	-2.01122\\
51.85	-2.02457\\
51.9	-2.03305\\
51.95	-2.03899\\
52	-2.04464\\
52.05	-2.05191\\
52.1	-2.06215\\
52.15	-2.0761\\
52.2	-2.09389\\
52.25	-2.11509\\
52.3	-2.1389\\
52.35	-2.16425\\
52.4	-2.19007\\
52.45	-2.21532\\
52.5	-2.2392\\
52.55	-2.26119\\
52.6	-2.28105\\
52.65	-2.29883\\
52.7	-2.31482\\
52.75	-2.32947\\
52.8	-2.3433\\
52.85	-2.35685\\
52.9	-2.37057\\
52.95	-2.38483\\
53	-2.39984\\
53.05	-2.416\\
53.1	-2.43571\\
53.15	-2.4623\\
53.2	-2.49796\\
53.25	-2.54278\\
53.3	-2.59461\\
53.35	-2.64968\\
53.4	-2.70345\\
53.45	-2.75155\\
53.5	-2.79057\\
53.55	-2.81859\\
53.6	-2.83537\\
53.65	-2.84222\\
53.7	-2.84167\\
53.75	-2.83692\\
53.8	-2.83136\\
53.85	-2.82799\\
53.9	-2.82914\\
53.95	-2.83624\\
54	-2.84975\\
54.05	-2.86926\\
54.1	-2.89374\\
54.15	-2.9217\\
54.2	-2.9515\\
54.25	-2.98158\\
54.3	-3.01062\\
54.35	-3.03766\\
54.4	-3.06217\\
54.45	-3.084\\
54.5	-3.10335\\
54.55	-3.12067\\
54.6	-3.13657\\
54.65	-3.15169\\
54.7	-3.16663\\
54.75	-3.18183\\
54.8	-3.1976\\
54.85	-3.21407\\
54.9	-3.23123\\
54.95	-3.24897\\
55	-3.26711\\
55.05	-3.27452\\
55.1	-3.24349\\
55.15	-3.16119\\
55.2	-3.02758\\
55.25	-2.85137\\
55.3	-2.64633\\
55.35	-2.42805\\
55.4	-2.21131\\
55.45	-2.00829\\
55.5	-1.8275\\
55.55	-1.67362\\
55.6	-1.5478\\
55.65	-1.44849\\
55.7	-1.37222\\
55.75	-1.31461\\
55.8	-1.27101\\
55.85	-1.23711\\
55.9	-1.20933\\
55.95	-1.18492\\
56	-1.16206\\
56.05	-1.13973\\
56.1	-1.11755\\
56.15	-1.09562\\
56.2	-1.07432\\
56.25	-1.05413\\
56.3	-1.03552\\
56.35	-1.01886\\
56.4	-1.00436\\
56.45	-0.99205\\
56.5	-0.98183\\
56.55	-0.97347\\
56.6	-0.96669\\
56.65	-0.96113\\
56.7	-0.956479999999999\\
56.75	-0.952439999999999\\
56.8	-0.948739999999999\\
56.85	-0.9452\\
56.9	-0.94168\\
56.95	-0.93809\\
57	-0.93436\\
57.05	-0.93049\\
57.1	-0.926449999999999\\
57.15	-0.92223\\
57.2	-0.91785\\
57.25	-0.913309999999999\\
57.3	-0.908689999999999\\
57.35	-0.904179999999999\\
57.4	-0.90006\\
57.45	-0.896579999999999\\
57.5	-0.89396\\
57.55	-0.89239\\
57.6	-0.891959999999999\\
57.65	-0.89272\\
57.7	-0.894659999999999\\
57.75	-0.89772\\
57.8	-0.901809999999999\\
57.85	-0.9068\\
57.9	-0.91259\\
57.95	-0.919049999999999\\
58	-0.92608\\
58.05	-0.93357\\
58.1	-0.941439999999999\\
58.15	-0.949629999999999\\
58.2	-0.958089999999999\\
58.25	-0.966779999999999\\
58.3	-0.97568\\
58.35	-0.984769999999999\\
58.4	-0.99403\\
58.45	-1.00344\\
58.5	-1.013\\
58.55	-1.02269\\
58.6	-1.03249\\
58.65	-1.04239\\
58.7	-1.05236\\
58.75	-1.06239\\
58.8	-1.07247\\
58.85	-1.08256\\
58.9	-1.09266\\
58.95	-1.10275\\
59	-1.11282\\
59.05	-1.12286\\
59.1	-1.13286\\
59.15	-1.1428\\
59.2	-1.15269\\
59.25	-1.16251\\
59.3	-1.17227\\
59.35	-1.18196\\
59.4	-1.19158\\
59.45	-1.20111\\
59.5	-1.21057\\
59.55	-1.21994\\
59.6	-1.22921\\
59.65	-1.2384\\
59.7	-1.24749\\
59.75	-1.25648\\
59.8	-1.26538\\
59.85	-1.27416\\
59.9	-1.28284\\
59.95	-1.29141\\
60	-1.29987\\
60.05	-1.30822\\
60.1	-1.31645\\
60.15	-1.32457\\
60.2	-1.33257\\
60.25	-1.34046\\
60.3	-1.34823\\
60.35	-1.35588\\
60.4	-1.36341\\
60.45	-1.37082\\
60.5	-1.37811\\
60.55	-1.38529\\
60.6	-1.39234\\
60.65	-1.39928\\
60.7	-1.4061\\
60.75	-1.41279\\
60.8	-1.41937\\
60.85	-1.42583\\
60.9	-1.43217\\
60.95	-1.4384\\
61	-1.4445\\
61.05	-1.45049\\
61.1	-1.45636\\
61.15	-1.46211\\
61.2	-1.46775\\
61.25	-1.47327\\
61.3	-1.47867\\
61.35	-1.48396\\
61.4	-1.48913\\
61.45	-1.49419\\
61.5	-1.49914\\
61.55	-1.50397\\
61.6	-1.50869\\
61.65	-1.5133\\
61.7	-1.5178\\
61.75	-1.52219\\
61.8	-1.52647\\
61.85	-1.53064\\
61.9	-1.5347\\
61.95	-1.53865\\
62	-1.54249\\
62.05	-1.54623\\
62.1	-1.54986\\
62.15	-1.55339\\
62.2	-1.55681\\
62.25	-1.56013\\
62.3	-1.56334\\
62.35	-1.56646\\
62.4	-1.56947\\
62.45	-1.57238\\
62.5	-1.57519\\
62.55	-1.57789\\
62.6	-1.5805\\
62.65	-1.58301\\
62.7	-1.58543\\
62.75	-1.58774\\
62.8	-1.58996\\
62.85	-1.59209\\
62.9	-1.59412\\
62.95	-1.59605\\
63	-1.5979\\
63.05	-1.59964\\
63.1	-1.6013\\
63.15	-1.60287\\
63.2	-1.60434\\
63.25	-1.60573\\
63.3	-1.60702\\
63.35	-1.60823\\
63.4	-1.60935\\
63.45	-1.61038\\
63.5	-1.61133\\
63.55	-1.61219\\
63.6	-1.61296\\
63.65	-1.61365\\
63.7	-1.61426\\
63.75	-1.61479\\
63.8	-1.61523\\
63.85	-1.61559\\
63.9	-1.61587\\
63.95	-1.61607\\
64	-1.61619\\
64.05	-1.61623\\
64.1	-1.6162\\
64.15	-1.61609\\
64.2	-1.6159\\
64.25	-1.61563\\
64.3	-1.61529\\
64.35	-1.61488\\
64.4	-1.61439\\
64.45	-1.61383\\
64.5	-1.61319\\
64.55	-1.61249\\
64.6	-1.61171\\
64.65	-1.61087\\
64.7	-1.60995\\
64.75	-1.60897\\
64.8	-1.60791\\
64.85	-1.60679\\
64.9	-1.60561\\
64.95	-1.60435\\
65	-1.60303\\
65.05	-1.60165\\
65.1	-1.6002\\
65.15	-1.59869\\
65.2	-1.59712\\
65.25	-1.59548\\
65.3	-1.59379\\
65.35	-1.59203\\
65.4	-1.59021\\
65.45	-1.58833\\
65.5	-1.5864\\
65.55	-1.5844\\
65.6	-1.58235\\
65.65	-1.58024\\
65.7	-1.57808\\
65.75	-1.57586\\
65.8	-1.57358\\
65.85	-1.57125\\
65.9	-1.56887\\
65.95	-1.56643\\
66	-1.56395\\
66.05	-1.56141\\
66.1	-1.55882\\
66.15	-1.55618\\
66.2	-1.55348\\
66.25	-1.55074\\
66.3	-1.54796\\
66.35	-1.54512\\
66.4	-1.54224\\
66.45	-1.53931\\
66.5	-1.53633\\
66.55	-1.53331\\
66.6	-1.53024\\
66.65	-1.52713\\
66.7	-1.52397\\
66.75	-1.52078\\
66.8	-1.51753\\
66.85	-1.51425\\
66.9	-1.51093\\
66.95	-1.50756\\
67	-1.50416\\
67.05	-1.50071\\
67.1	-1.49723\\
67.15	-1.49371\\
67.2	-1.49015\\
67.25	-1.48655\\
67.3	-1.48291\\
67.35	-1.47924\\
67.4	-1.47554\\
67.45	-1.47179\\
67.5	-1.46802\\
67.55	-1.4642\\
67.6	-1.46036\\
67.65	-1.45648\\
67.7	-1.45257\\
67.75	-1.44863\\
67.8	-1.44465\\
67.85	-1.44065\\
67.9	-1.43661\\
67.95	-1.43255\\
68	-1.42845\\
68.05	-1.42433\\
68.1	-1.42017\\
68.15	-1.41599\\
68.2	-1.41178\\
68.25	-1.40754\\
68.3	-1.40328\\
68.35	-1.39899\\
68.4	-1.39468\\
68.45	-1.39034\\
68.5	-1.38597\\
68.55	-1.38158\\
68.6	-1.37717\\
68.65	-1.37273\\
68.7	-1.36827\\
68.75	-1.36379\\
68.8	-1.35928\\
68.85	-1.35475\\
68.9	-1.35021\\
68.95	-1.34564\\
69	-1.34105\\
69.05	-1.33644\\
69.1	-1.33181\\
69.15	-1.32716\\
69.2	-1.32249\\
69.25	-1.31781\\
69.3	-1.3131\\
69.35	-1.30838\\
69.4	-1.30364\\
69.45	-1.29888\\
69.5	-1.29411\\
69.55	-1.28932\\
69.6	-1.28452\\
69.65	-1.2797\\
69.7	-1.27486\\
69.75	-1.27001\\
69.8	-1.26515\\
69.85	-1.26027\\
69.9	-1.25538\\
69.95	-1.25047\\
70	-1.24555\\
70.05	-1.24062\\
70.1	-1.23568\\
70.15	-1.23073\\
70.2	-1.22576\\
70.25	-1.22078\\
70.3	-1.21579\\
70.35	-1.21079\\
70.4	-1.20578\\
70.45	-1.20076\\
70.5	-1.19573\\
70.55	-1.19069\\
70.6	-1.18564\\
70.65	-1.18058\\
70.7	-1.17552\\
70.75	-1.17044\\
70.8	-1.16536\\
70.85	-1.16027\\
70.9	-1.15517\\
70.95	-1.15007\\
71	-1.14496\\
71.05	-1.13984\\
71.1	-1.13471\\
71.15	-1.12958\\
71.2	-1.12444\\
71.25	-1.1193\\
71.3	-1.11415\\
71.35	-1.109\\
71.4	-1.10384\\
71.45	-1.09868\\
71.5	-1.09351\\
71.55	-1.08834\\
71.6	-1.08317\\
71.65	-1.07799\\
71.7	-1.07281\\
71.75	-1.06762\\
71.8	-1.06243\\
71.85	-1.05724\\
71.9	-1.05205\\
71.95	-1.04685\\
72	-1.04165\\
72.05	-1.03645\\
72.1	-1.03125\\
72.15	-1.02605\\
72.2	-1.02084\\
72.25	-1.01564\\
72.3	-1.01043\\
72.35	-1.00522\\
72.4	-1.00001\\
72.45	-0.9948\\
72.5	-0.98959\\
72.55	-0.984389999999999\\
72.6	-0.979179999999999\\
72.65	-0.97397\\
72.7	-0.96876\\
72.75	-0.96355\\
72.8	-0.958349999999999\\
72.85	-0.953139999999999\\
72.9	-0.947939999999999\\
72.95	-0.94274\\
73	-0.937539999999999\\
73.05	-0.932339999999999\\
73.1	-0.92714\\
73.15	-0.921939999999999\\
73.2	-0.91675\\
73.25	-0.91156\\
73.3	-0.906369999999999\\
73.35	-0.901179999999999\\
73.4	-0.895999999999999\\
73.45	-0.89082\\
73.5	-0.88564\\
73.55	-0.88047\\
73.6	-0.875299999999999\\
73.65	-0.87013\\
73.7	-0.86497\\
73.75	-0.85981\\
73.8	-0.854649999999999\\
73.85	-0.849499999999999\\
73.9	-0.844349999999999\\
73.95	-0.83921\\
74	-0.83407\\
74.05	-0.82893\\
74.1	-0.823799999999999\\
74.15	-0.818669999999999\\
74.2	-0.813549999999999\\
74.25	-0.808439999999999\\
74.3	-0.80333\\
74.35	-0.79822\\
74.4	-0.793119999999999\\
74.45	-0.788019999999999\\
74.5	-0.782929999999999\\
74.55	-0.777849999999999\\
74.6	-0.77277\\
74.65	-0.7677\\
74.7	-0.76263\\
74.75	-0.757569999999999\\
74.8	-0.752509999999999\\
74.85	-0.747459999999999\\
74.9	-0.742419999999999\\
74.95	-0.737379999999999\\
75	-0.732349999999999\\
75.05	-0.727329999999999\\
75.1	-0.722309999999999\\
75.15	-0.7173\\
75.2	-0.712299999999999\\
75.25	-0.707299999999999\\
75.3	-0.70231\\
75.35	-0.697329999999999\\
75.4	-0.69236\\
75.45	-0.68739\\
75.5	-0.682429999999999\\
75.55	-0.67747\\
75.6	-0.672529999999999\\
75.65	-0.66759\\
75.7	-0.66266\\
75.75	-0.657729999999999\\
75.8	-0.652819999999999\\
75.85	-0.64791\\
75.9	-0.643009999999999\\
75.95	-0.63812\\
76	-0.633229999999999\\
76.05	-0.62836\\
76.1	-0.623489999999999\\
76.15	-0.61863\\
76.2	-0.613779999999999\\
76.25	-0.60894\\
76.3	-0.6041\\
76.35	-0.59927\\
76.4	-0.59446\\
76.45	-0.58965\\
76.5	-0.584849999999999\\
76.55	-0.580049999999999\\
76.6	-0.57527\\
76.65	-0.570499999999999\\
76.7	-0.565729999999999\\
76.75	-0.560969999999999\\
76.8	-0.55622\\
76.85	-0.551489999999999\\
76.9	-0.546759999999999\\
76.95	-0.54203\\
77	-0.537319999999999\\
77.05	-0.53262\\
77.1	-0.52793\\
77.15	-0.523239999999999\\
77.2	-0.51857\\
77.25	-0.5139\\
77.3	-0.50925\\
77.35	-0.504599999999999\\
77.4	-0.49996\\
77.45	-0.49534\\
77.5	-0.49072\\
77.55	-0.486109999999999\\
77.6	-0.481509999999999\\
77.65	-0.47692\\
77.7	-0.472339999999999\\
77.75	-0.46777\\
77.8	-0.463209999999999\\
77.85	-0.458659999999999\\
77.9	-0.45412\\
77.95	-0.44959\\
78	-0.445069999999999\\
78.05	-0.44056\\
78.1	-0.436059999999999\\
78.15	-0.43157\\
78.2	-0.42709\\
78.25	-0.422619999999999\\
78.3	-0.418159999999999\\
78.35	-0.413709999999999\\
78.4	-0.40928\\
78.45	-0.40485\\
78.5	-0.400429999999999\\
78.55	-0.396019999999999\\
78.6	-0.39162\\
78.65	-0.38723\\
78.7	-0.382859999999999\\
78.75	-0.378489999999999\\
78.8	-0.374129999999999\\
78.85	-0.369789999999999\\
78.9	-0.365449999999999\\
78.95	-0.361129999999999\\
79	-0.356809999999999\\
79.05	-0.35251\\
79.1	-0.34822\\
79.15	-0.343929999999999\\
79.2	-0.339659999999999\\
79.25	-0.335399999999999\\
79.3	-0.331149999999999\\
79.35	-0.32691\\
79.4	-0.322679999999999\\
79.45	-0.318459999999999\\
79.5	-0.314249999999999\\
79.55	-0.310059999999999\\
79.6	-0.30587\\
79.65	-0.30169\\
79.7	-0.297529999999999\\
79.75	-0.293379999999999\\
79.8	-0.28923\\
79.85	-0.285099999999999\\
79.9	-0.28098\\
79.95	-0.27687\\
80	-0.27277\\
};
\end{axis}
\begin{axis}[%
width=5cm,
height=4cm,
at={(6cm,0\linewidth)},
scale only axis,
xmin=50,
xmax=80,
xlabel={Time [s]},
xmajorgrids,
ymin=-6,
ymax=1,
% ylabel={Relative Surge Control Distance [\%]},
ymajorgrids,
axis background/.style={fill=white},
title style={font=\bfseries},
title={Upstream Compressor},
axis x line*=bottom,
axis y line*=left
]
\addplot [color=mycolor1,solid,line width=1.5pt,forget plot]
  table[row sep=crcr]{%
50	-0.00287999999999933\\
50.05	-0.00287999999999933\\
50.1	-0.340039999999999\\
50.15	-1.01667\\
50.2	-1.8749\\
50.25	-2.77916\\
50.3	-3.61056\\
50.35	-4.28728\\
50.4	-4.76714\\
50.45	-5.02619\\
50.5	-5.05865\\
50.55	-4.88044\\
50.6	-4.53806\\
50.65	-4.10178\\
50.7	-3.6449\\
50.75	-3.23165\\
50.8	-2.90792\\
50.85	-2.69721\\
50.9	-2.6018\\
50.95	-2.60755\\
51	-2.68995\\
51.05	-2.82\\
51.1	-2.96888\\
51.15	-3.11134\\
51.2	-3.22777\\
51.25	-3.30532\\
51.3	-3.33818\\
51.35	-3.32708\\
51.4	-3.27823\\
51.45	-3.20171\\
51.5	-3.10962\\
51.55	-3.01412\\
51.6	-2.92583\\
51.65	-2.85258\\
51.7	-2.79888\\
51.75	-2.76589\\
51.8	-2.75192\\
51.85	-2.75319\\
51.9	-2.7647\\
51.95	-2.78114\\
52	-2.79788\\
52.05	-2.81261\\
52.1	-2.82479\\
52.15	-2.83436\\
52.2	-2.84165\\
52.25	-2.84727\\
52.3	-2.84035\\
52.35	-2.78848\\
52.4	-2.67211\\
52.45	-2.48642\\
52.5	-2.23862\\
52.55	-1.94421\\
52.6	-1.62329\\
52.65	-1.29773\\
52.7	-0.991169999999999\\
52.75	-0.72666\\
52.8	-0.520689999999999\\
52.85	-0.38019\\
52.9	-0.30247\\
52.95	-0.27718\\
53	-0.289249999999999\\
53.05	-0.32201\\
53.1	-0.35976\\
53.15	-0.389729999999999\\
53.2	-0.403169999999999\\
53.25	-0.39575\\
53.3	-0.36727\\
53.35	-0.32081\\
53.4	-0.261649999999999\\
53.45	-0.195969999999999\\
53.5	-0.128729999999999\\
53.55	-0.0633299999999988\\
53.6	-0.00231999999999921\\
53.65	0.0517599999999998\\
53.7	0.0982300000000009\\
53.75	0.138570000000001\\
53.8	0.174570000000001\\
53.85	0.208080000000001\\
53.9	0.240860000000001\\
53.95	0.27434\\
54	0.309560000000001\\
54.05	0.34709\\
54.1	0.387090000000001\\
54.15	0.429410000000001\\
54.2	0.473650000000001\\
54.25	0.51932\\
54.3	0.56588\\
54.35	0.612870000000001\\
54.4	0.659830000000001\\
54.45	0.70623\\
54.5	0.75141\\
54.55	0.79461\\
54.6	0.83497\\
54.65	0.871540000000001\\
54.7	0.903370000000001\\
54.75	0.92952\\
54.8	0.94922\\
54.85	0.96194\\
54.9	0.967500000000001\\
54.95	0.966090000000001\\
55	0.958270000000001\\
55.05	0.94492\\
55.1	0.927160000000001\\
55.15	0.90619\\
55.2	0.883240000000001\\
55.25	0.85941\\
55.3	0.83564\\
55.35	0.812610000000001\\
55.4	0.79078\\
55.45	0.76999\\
55.5	0.747640000000001\\
55.55	0.72021\\
55.6	0.6861\\
55.65	0.648050000000001\\
55.7	0.60947\\
55.75	0.572740000000001\\
55.8	0.539760000000001\\
55.85	0.512170000000001\\
55.9	0.491190000000001\\
55.95	0.477500000000001\\
56	0.471160000000001\\
56.05	0.471530000000001\\
56.1	0.477400000000001\\
56.15	0.48715\\
56.2	0.499000000000001\\
56.25	0.511200000000001\\
56.3	0.522250000000001\\
56.35	0.531040000000001\\
56.4	0.53688\\
56.45	0.539520000000001\\
56.5	0.53909\\
56.55	0.536010000000001\\
56.6	0.53087\\
56.65	0.52435\\
56.7	0.517110000000001\\
56.75	0.50972\\
56.8	0.50263\\
56.85	0.49614\\
56.9	0.49042\\
56.95	0.4855\\
57	0.481300000000001\\
57.05	0.4777\\
57.1	0.474580000000001\\
57.15	0.47184\\
57.2	0.469390000000001\\
57.25	0.467120000000001\\
57.3	0.464930000000001\\
57.35	0.462730000000001\\
57.4	0.4604\\
57.45	0.45782\\
57.5	0.45491\\
57.55	0.45158\\
57.6	0.447790000000001\\
57.65	0.44356\\
57.7	0.43895\\
57.75	0.434140000000001\\
57.8	0.429310000000001\\
57.85	0.424710000000001\\
57.9	0.420580000000001\\
57.95	0.417160000000001\\
58	0.41464\\
58.05	0.413180000000001\\
58.1	0.412830000000001\\
58.15	0.41361\\
58.2	0.415460000000001\\
58.25	0.41826\\
58.3	0.421840000000001\\
58.35	0.426020000000001\\
58.4	0.43061\\
58.45	0.43544\\
58.5	0.440330000000001\\
58.55	0.445180000000001\\
58.6	0.44988\\
58.65	0.45438\\
58.7	0.458640000000001\\
58.75	0.462670000000001\\
58.8	0.466470000000001\\
58.85	0.47007\\
58.9	0.47349\\
58.95	0.476750000000001\\
59	0.479850000000001\\
59.05	0.482810000000001\\
59.1	0.485610000000001\\
59.15	0.488240000000001\\
59.2	0.490680000000001\\
59.25	0.49292\\
59.3	0.494910000000001\\
59.35	0.496650000000001\\
59.4	0.49812\\
59.45	0.49929\\
59.5	0.500150000000001\\
59.55	0.500720000000001\\
59.6	0.500970000000001\\
59.65	0.500910000000001\\
59.7	0.50056\\
59.75	0.499920000000001\\
59.8	0.498990000000001\\
59.85	0.49779\\
59.9	0.49633\\
59.95	0.494630000000001\\
60	0.49268\\
60.05	0.490490000000001\\
60.1	0.48808\\
60.15	0.48545\\
60.2	0.482610000000001\\
60.25	0.47955\\
60.3	0.476290000000001\\
60.35	0.47282\\
60.4	0.469150000000001\\
60.45	0.46528\\
60.5	0.461220000000001\\
60.55	0.45697\\
60.6	0.45252\\
60.65	0.447900000000001\\
60.7	0.443100000000001\\
60.75	0.438130000000001\\
60.8	0.43299\\
60.85	0.4277\\
60.9	0.422260000000001\\
60.95	0.41667\\
61	0.410960000000001\\
61.05	0.405110000000001\\
61.1	0.399140000000001\\
61.15	0.393050000000001\\
61.2	0.38686\\
61.25	0.380560000000001\\
61.3	0.37416\\
61.35	0.36767\\
61.4	0.361090000000001\\
61.45	0.354420000000001\\
61.5	0.347670000000001\\
61.55	0.340850000000001\\
61.6	0.33395\\
61.65	0.32699\\
61.7	0.31996\\
61.75	0.31287\\
61.8	0.305720000000001\\
61.85	0.298530000000001\\
61.9	0.29129\\
61.95	0.284000000000001\\
62	0.276670000000001\\
62.05	0.269310000000001\\
62.1	0.26192\\
62.15	0.2545\\
62.2	0.24705\\
62.25	0.23958\\
62.3	0.23208\\
62.35	0.224580000000001\\
62.4	0.21706\\
62.45	0.209530000000001\\
62.5	0.20199\\
62.55	0.19445\\
62.6	0.186910000000001\\
62.65	0.17937\\
62.7	0.17183\\
62.75	0.16431\\
62.8	0.156790000000001\\
62.85	0.149290000000001\\
62.9	0.1418\\
62.95	0.13434\\
63	0.126890000000001\\
63.05	0.11947\\
63.1	0.112080000000001\\
63.15	0.10472\\
63.2	0.0974000000000004\\
63.25	0.090110000000001\\
63.3	0.0828600000000002\\
63.35	0.0756500000000013\\
63.4	0.0684900000000006\\
63.45	0.0613799999999998\\
63.5	0.054310000000001\\
63.55	0.0472999999999999\\
63.6	0.0403500000000001\\
63.65	0.0334599999999998\\
63.7	0.0266200000000012\\
63.75	0.0198499999999999\\
63.8	0.0131500000000013\\
63.85	0.00652000000000008\\
63.9	-4.0000000000262e-05\\
63.95	-0.00652000000000008\\
64	-0.012929999999999\\
64.05	-0.0192599999999992\\
64.1	-0.0255099999999988\\
64.15	-0.0316700000000001\\
64.2	-0.0377399999999994\\
64.25	-0.0437199999999986\\
64.3	-0.0496099999999995\\
64.35	-0.0554100000000002\\
64.4	-0.0610999999999997\\
64.45	-0.0666999999999991\\
64.5	-0.0721999999999987\\
64.55	-0.0775899999999989\\
64.6	-0.0828699999999998\\
64.65	-0.0880499999999991\\
64.7	-0.0931099999999994\\
64.75	-0.0980600000000003\\
64.8	-0.102889999999999\\
64.85	-0.1076\\
64.9	-0.11219\\
64.95	-0.11666\\
65	-0.120999999999999\\
65.05	-0.125209999999999\\
65.1	-0.1293\\
65.15	-0.13324\\
65.2	-0.137059999999999\\
65.25	-0.14073\\
65.3	-0.144259999999999\\
65.35	-0.147659999999999\\
65.4	-0.150899999999999\\
65.45	-0.153999999999999\\
65.5	-0.15696\\
65.55	-0.159759999999999\\
65.6	-0.162409999999999\\
65.65	-0.164909999999999\\
65.7	-0.16726\\
65.75	-0.169449999999999\\
65.8	-0.171489999999999\\
65.85	-0.17338\\
65.9	-0.17512\\
65.95	-0.176699999999999\\
66	-0.178139999999999\\
66.05	-0.179429999999999\\
66.1	-0.180579999999999\\
66.15	-0.181589999999999\\
66.2	-0.18246\\
66.25	-0.183199999999999\\
66.3	-0.183809999999999\\
66.35	-0.184309999999999\\
66.4	-0.184679999999999\\
66.45	-0.184939999999999\\
66.5	-0.185099999999999\\
66.55	-0.185149999999999\\
66.6	-0.18511\\
66.65	-0.18497\\
66.7	-0.18474\\
66.75	-0.184419999999999\\
66.8	-0.18401\\
66.85	-0.18352\\
66.9	-0.182939999999999\\
66.95	-0.182289999999999\\
67	-0.181539999999999\\
67.05	-0.180719999999999\\
67.1	-0.179819999999999\\
67.15	-0.17883\\
67.2	-0.177759999999999\\
67.25	-0.176609999999999\\
67.3	-0.17538\\
67.35	-0.17407\\
67.4	-0.172689999999999\\
67.45	-0.171219999999999\\
67.5	-0.16968\\
67.55	-0.16806\\
67.6	-0.16637\\
67.65	-0.16461\\
67.7	-0.16278\\
67.75	-0.160889999999999\\
67.8	-0.15893\\
67.85	-0.156899999999999\\
67.9	-0.154819999999999\\
67.95	-0.152679999999999\\
68	-0.15049\\
68.05	-0.148239999999999\\
68.1	-0.145949999999999\\
68.15	-0.14361\\
68.2	-0.141229999999999\\
68.25	-0.138809999999999\\
68.3	-0.136349999999999\\
68.35	-0.13385\\
68.4	-0.131329999999999\\
68.45	-0.128769999999999\\
68.5	-0.126189999999999\\
68.55	-0.12359\\
68.6	-0.12097\\
68.65	-0.11833\\
68.7	-0.115679999999999\\
68.75	-0.113019999999999\\
68.8	-0.110349999999999\\
68.85	-0.107669999999999\\
68.9	-0.104989999999999\\
68.95	-0.102309999999999\\
69	-0.0996299999999994\\
69.05	-0.0969599999999993\\
69.1	-0.0942899999999991\\
69.15	-0.0916399999999999\\
69.2	-0.088989999999999\\
69.25	-0.0863599999999991\\
69.3	-0.0837399999999988\\
69.35	-0.0811399999999995\\
69.4	-0.0785699999999991\\
69.45	-0.0760100000000001\\
69.5	-0.0734699999999986\\
69.55	-0.0709599999999995\\
69.6	-0.0684799999999992\\
69.65	-0.0660299999999996\\
69.7	-0.0635999999999992\\
69.75	-0.0612099999999991\\
69.8	-0.0588499999999996\\
69.85	-0.056519999999999\\
69.9	-0.054219999999999\\
69.95	-0.0519599999999993\\
70	-0.0497399999999999\\
70.05	-0.0475599999999989\\
70.1	-0.0454099999999986\\
70.15	-0.0432999999999986\\
70.2	-0.0412400000000002\\
70.25	-0.0392099999999989\\
70.3	-0.0372199999999996\\
70.35	-0.0352800000000002\\
70.4	-0.0333799999999993\\
70.45	-0.0315199999999987\\
70.5	-0.0297000000000001\\
70.55	-0.0279299999999996\\
70.6	-0.0261999999999993\\
70.65	-0.0245099999999994\\
70.7	-0.0228599999999997\\
70.75	-0.0212699999999995\\
70.8	-0.0197099999999999\\
70.85	-0.0182000000000002\\
70.9	-0.016729999999999\\
70.95	-0.0152999999999999\\
71	-0.0139199999999988\\
71.05	-0.0125799999999998\\
71.1	-0.0112899999999989\\
71.15	-0.01004\\
71.2	-0.00882999999999967\\
71.25	-0.00765999999999956\\
71.3	-0.00653999999999932\\
71.35	-0.00544999999999973\\
71.4	-0.00441000000000003\\
71.45	-0.0034099999999988\\
71.5	-0.00244999999999962\\
71.55	-0.00152999999999892\\
71.6	-0.000650000000000261\\
71.65	0.000189999999999912\\
71.7	0.00100000000000122\\
71.75	0.00176000000000087\\
71.8	0.00248999999999988\\
71.85	0.0031800000000004\\
71.9	0.00384000000000029\\
71.95	0.00445999999999991\\
72	0.00505000000000067\\
72.05	0.00560000000000116\\
72.1	0.00611000000000139\\
72.15	0.00660000000000061\\
72.2	0.00705000000000133\\
72.25	0.00747000000000142\\
72.3	0.00786000000000087\\
72.35	0.00821999999999967\\
72.4	0.00853999999999999\\
72.45	0.00884000000000107\\
72.5	0.00912000000000113\\
72.55	0.00936000000000092\\
72.6	0.0095799999999997\\
72.65	0.00977000000000139\\
72.7	0.00993000000000066\\
72.75	0.0100700000000007\\
72.8	0.0101899999999997\\
72.85	0.0102799999999998\\
72.9	0.0103500000000007\\
72.95	0.010390000000001\\
73	0.0104199999999999\\
73.05	0.0104199999999999\\
73.1	0.0104100000000003\\
73.15	0.01037\\
73.2	0.0103200000000001\\
73.25	0.0102400000000014\\
73.3	0.0101500000000012\\
73.35	0.01004\\
73.4	0.00992000000000104\\
73.45	0.00978000000000101\\
73.5	0.00961999999999996\\
73.55	0.00945000000000107\\
73.6	0.00927000000000078\\
73.65	0.00907000000000124\\
73.7	0.00886000000000031\\
73.75	0.00863999999999976\\
73.8	0.00839999999999996\\
73.85	0.00816000000000017\\
73.9	0.00790000000000113\\
73.95	0.00763000000000069\\
74	0.00735000000000063\\
74.05	0.00707000000000058\\
74.1	0.00677000000000127\\
74.15	0.0064700000000002\\
74.2	0.00616000000000128\\
74.25	0.00584000000000096\\
74.3	0.00551000000000101\\
74.35	0.00518000000000107\\
74.4	0.00483999999999973\\
74.45	0.00450000000000017\\
74.5	0.00415000000000099\\
74.55	0.00380000000000003\\
74.6	0.00344000000000122\\
74.65	0.00308000000000064\\
74.7	0.00271000000000043\\
74.75	0.00234000000000023\\
74.8	0.00197000000000003\\
74.85	0.0015900000000002\\
74.9	0.00122\\
74.95	0.000840000000000174\\
75	0.000460000000000349\\
75.05	7.00000000009027e-05\\
75.1	-0.000309999999998922\\
75.15	-0.000700000000000145\\
75.2	-0.00107999999999997\\
75.25	-0.00146999999999942\\
75.3	-0.00185999999999886\\
75.35	-0.00223999999999869\\
75.4	-0.00262999999999991\\
75.45	-0.00301999999999936\\
75.5	-0.00339999999999918\\
75.55	-0.00378999999999863\\
75.6	-0.00417000000000023\\
75.65	-0.00455000000000005\\
75.7	-0.0049399999999995\\
75.75	-0.00531999999999933\\
75.8	-0.00568999999999953\\
75.85	-0.00606999999999935\\
75.9	-0.00643999999999956\\
75.95	-0.00681999999999938\\
76	-0.00718999999999959\\
76.05	-0.00755999999999979\\
76.1	-0.00791999999999859\\
76.15	-0.00827999999999918\\
76.2	-0.00863999999999976\\
76.25	-0.00899999999999856\\
76.3	-0.00935999999999915\\
76.35	-0.00971000000000011\\
76.4	-0.0100599999999993\\
76.45	-0.0104100000000003\\
76.5	-0.0107499999999998\\
76.55	-0.0110899999999994\\
76.6	-0.0114299999999989\\
76.65	-0.0117599999999989\\
76.7	-0.0120899999999988\\
76.75	-0.0124199999999988\\
76.8	-0.0127399999999991\\
76.85	-0.0130599999999994\\
76.9	-0.0133799999999997\\
76.95	-0.0137\\
77	-0.014009999999999\\
77.05	-0.0143199999999997\\
77.1	-0.014619999999999\\
77.15	-0.01492\\
77.2	-0.0152199999999993\\
77.25	-0.015509999999999\\
77.3	-0.0158100000000001\\
77.35	-0.0160900000000002\\
77.4	-0.0163799999999998\\
77.45	-0.0166599999999999\\
77.5	-0.01694\\
77.55	-0.0172099999999986\\
77.6	-0.0174799999999991\\
77.65	-0.0177499999999995\\
77.7	-0.0180199999999999\\
77.75	-0.018279999999999\\
77.8	-0.0185399999999998\\
77.85	-0.0187999999999988\\
77.9	-0.01905\\
77.95	-0.0192999999999994\\
78	-0.0195499999999988\\
78.05	-0.0198\\
78.1	-0.0200399999999998\\
78.15	-0.0202799999999996\\
78.2	-0.0205099999999998\\
78.25	-0.0207499999999996\\
78.3	-0.0209799999999998\\
78.35	-0.02121\\
78.4	-0.0214400000000001\\
78.45	-0.0216599999999989\\
78.5	-0.0218799999999995\\
78.55	-0.0221\\
78.6	-0.0223199999999988\\
78.65	-0.0225399999999993\\
78.7	-0.0227500000000003\\
78.75	-0.0229599999999994\\
78.8	-0.0231699999999986\\
78.85	-0.0233799999999995\\
78.9	-0.023579999999999\\
78.95	-0.0237799999999986\\
79	-0.0239899999999995\\
79.05	-0.024189999999999\\
79.1	-0.024379999999999\\
79.15	-0.0245800000000003\\
79.2	-0.0247799999999998\\
79.25	-0.0249699999999997\\
79.3	-0.0251599999999996\\
79.35	-0.0253499999999995\\
79.4	-0.0255399999999995\\
79.45	-0.0257299999999994\\
79.5	-0.0259099999999997\\
79.55	-0.0260999999999996\\
79.6	-0.0262799999999999\\
79.65	-0.0264699999999998\\
79.7	-0.0266500000000001\\
79.75	-0.0268299999999986\\
79.8	-0.0270099999999989\\
79.85	-0.0271899999999992\\
79.9	-0.0273699999999995\\
79.95	-0.0275499999999997\\
80	-0.0277199999999986\\
};
\addplot [color=mycolor2,dashed,line width=1.5pt,forget plot]
  table[row sep=crcr]{%
50	-0.00324999999999953\\
50.05	-0.00324999999999953\\
50.1	-0.34031\\
50.15	-1.01684\\
50.2	-1.87543\\
50.25	-2.78574\\
50.3	-3.64011\\
50.35	-4.3476\\
50.4	-4.83738\\
50.45	-5.08048\\
50.5	-5.08938\\
50.55	-4.89977\\
50.6	-4.56668\\
50.65	-4.15677\\
50.7	-3.73755\\
50.75	-3.36652\\
50.8	-3.0835\\
50.85	-2.90762\\
50.9	-2.83882\\
50.95	-2.86244\\
51	-2.95481\\
51.05	-3.08852\\
51.1	-3.23664\\
51.15	-3.37579\\
51.2	-3.48803\\
51.25	-3.56188\\
51.3	-3.59251\\
51.35	-3.58127\\
51.4	-3.53458\\
51.45	-3.46241\\
51.5	-3.37647\\
51.55	-3.28834\\
51.6	-3.20802\\
51.65	-3.14275\\
51.7	-3.09658\\
51.75	-3.07036\\
51.8	-3.06226\\
51.85	-3.06845\\
51.9	-3.08402\\
51.95	-3.10378\\
52	-3.12297\\
52.05	-3.13776\\
52.1	-3.14559\\
52.15	-3.14529\\
52.2	-3.13697\\
52.25	-3.12183\\
52.3	-3.1018\\
52.35	-3.07923\\
52.4	-3.04421\\
52.45	-2.96383\\
52.5	-2.81816\\
52.55	-2.6027\\
52.6	-2.32609\\
52.65	-2.00638\\
52.7	-1.66886\\
52.75	-1.34351\\
52.8	-1.0586\\
52.85	-0.83456\\
52.9	-0.680859999999999\\
52.95	-0.595949999999999\\
53	-0.569399999999999\\
53.05	-0.585179999999999\\
53.1	-0.62505\\
53.15	-0.671469999999999\\
53.2	-0.709829999999999\\
53.25	-0.72974\\
53.3	-0.725639999999999\\
53.35	-0.69659\\
53.4	-0.64556\\
53.45	-0.57823\\
53.5	-0.501639999999999\\
53.55	-0.42282\\
53.6	-0.3477\\
53.65	-0.280399999999999\\
53.7	-0.22289\\
53.75	-0.175129999999999\\
53.8	-0.135479999999999\\
53.85	-0.101239999999999\\
53.9	-0.0692799999999991\\
53.95	-0.0365500000000001\\
54	-0.00672999999999924\\
54.05	0.0190999999999999\\
54.1	0.0428800000000003\\
54.15	0.0660000000000007\\
54.2	0.0922300000000007\\
54.25	0.124470000000001\\
54.3	0.1639\\
54.35	0.20978\\
54.4	0.260670000000001\\
54.45	0.31479\\
54.5	0.37031\\
54.55	0.42559\\
54.6	0.47926\\
54.65	0.53026\\
54.7	0.57779\\
54.75	0.62124\\
54.8	0.660170000000001\\
54.85	0.6943\\
54.9	0.72348\\
54.95	0.747710000000001\\
55	0.767150000000001\\
55.05	0.782110000000001\\
55.1	0.79302\\
55.15	0.800420000000001\\
55.2	0.80489\\
55.25	0.8071\\
55.3	0.807740000000001\\
55.35	0.807370000000001\\
55.4	0.80644\\
55.45	0.80522\\
55.5	0.803840000000001\\
55.55	0.802340000000001\\
55.6	0.800650000000001\\
55.65	0.79866\\
55.7	0.796240000000001\\
55.75	0.79331\\
55.8	0.789810000000001\\
55.85	0.785740000000001\\
55.9	0.78115\\
55.95	0.77576\\
56	0.76746\\
56.05	0.755190000000001\\
56.1	0.738000000000001\\
56.15	0.71494\\
56.2	0.68679\\
56.25	0.657780000000001\\
56.3	0.631740000000001\\
56.35	0.610480000000001\\
56.4	0.59465\\
56.45	0.584200000000001\\
56.5	0.57873\\
56.55	0.57752\\
56.6	0.57967\\
56.65	0.58417\\
56.7	0.589920000000001\\
56.75	0.59585\\
56.8	0.60102\\
56.85	0.60464\\
56.9	0.606150000000001\\
56.95	0.60525\\
57	0.601840000000001\\
57.05	0.596070000000001\\
57.1	0.58821\\
57.15	0.57863\\
57.2	0.567730000000001\\
57.25	0.555910000000001\\
57.3	0.543470000000001\\
57.35	0.530660000000001\\
57.4	0.517700000000001\\
57.45	0.50474\\
57.5	0.49188\\
57.55	0.47912\\
57.6	0.46644\\
57.65	0.453720000000001\\
57.7	0.440860000000001\\
57.75	0.427710000000001\\
57.8	0.414160000000001\\
57.85	0.400120000000001\\
57.9	0.38551\\
57.95	0.370320000000001\\
58	0.354560000000001\\
58.05	0.338290000000001\\
58.1	0.32159\\
58.15	0.304590000000001\\
58.2	0.28745\\
58.25	0.27036\\
58.3	0.25356\\
58.35	0.23728\\
58.4	0.22176\\
58.45	0.2072\\
58.5	0.193800000000001\\
58.55	0.18167\\
58.6	0.17088\\
58.65	0.161470000000001\\
58.7	0.15339\\
58.75	0.14659\\
58.8	0.14095\\
58.85	0.13636\\
58.9	0.132680000000001\\
58.95	0.12979\\
59	0.127550000000001\\
59.05	0.12588\\
59.1	0.12468\\
59.15	0.12388\\
59.2	0.12346\\
59.25	0.123380000000001\\
59.3	0.12364\\
59.35	0.124230000000001\\
59.4	0.125170000000001\\
59.45	0.12645\\
59.5	0.128080000000001\\
59.55	0.13007\\
59.6	0.13242\\
59.65	0.135110000000001\\
59.7	0.138120000000001\\
59.75	0.141440000000001\\
59.8	0.14504\\
59.85	0.14888\\
59.9	0.152950000000001\\
59.95	0.157220000000001\\
60	0.16165\\
60.05	0.166220000000001\\
60.1	0.170920000000001\\
60.15	0.17573\\
60.2	0.18064\\
60.25	0.185640000000001\\
60.3	0.190720000000001\\
60.35	0.195870000000001\\
60.4	0.2011\\
60.45	0.2064\\
60.5	0.21176\\
60.55	0.217180000000001\\
60.6	0.22265\\
60.65	0.228160000000001\\
60.7	0.233700000000001\\
60.75	0.23925\\
60.8	0.244810000000001\\
60.85	0.250360000000001\\
60.9	0.25587\\
60.95	0.26135\\
61	0.266760000000001\\
61.05	0.27209\\
61.1	0.277330000000001\\
61.15	0.28246\\
61.2	0.28748\\
61.25	0.29236\\
61.3	0.29711\\
61.35	0.30171\\
61.4	0.306150000000001\\
61.45	0.310420000000001\\
61.5	0.314530000000001\\
61.55	0.31846\\
61.6	0.32221\\
61.65	0.32578\\
61.7	0.329170000000001\\
61.75	0.332360000000001\\
61.8	0.335360000000001\\
61.85	0.33817\\
61.9	0.340780000000001\\
61.95	0.34318\\
62	0.34539\\
62.05	0.3474\\
62.1	0.349210000000001\\
62.15	0.350820000000001\\
62.2	0.35223\\
62.25	0.353440000000001\\
62.3	0.354460000000001\\
62.35	0.35528\\
62.4	0.35591\\
62.45	0.356350000000001\\
62.5	0.35661\\
62.55	0.356680000000001\\
62.6	0.356570000000001\\
62.65	0.356290000000001\\
62.7	0.355840000000001\\
62.75	0.355210000000001\\
62.8	0.354420000000001\\
62.85	0.35347\\
62.9	0.352360000000001\\
62.95	0.351090000000001\\
63	0.34967\\
63.05	0.348100000000001\\
63.1	0.346390000000001\\
63.15	0.34454\\
63.2	0.342560000000001\\
63.25	0.340440000000001\\
63.3	0.338190000000001\\
63.35	0.33581\\
63.4	0.333320000000001\\
63.45	0.33071\\
63.5	0.32799\\
63.55	0.325150000000001\\
63.6	0.32222\\
63.65	0.319180000000001\\
63.7	0.316040000000001\\
63.75	0.312810000000001\\
63.8	0.30949\\
63.85	0.30608\\
63.9	0.30259\\
63.95	0.299020000000001\\
64	0.29538\\
64.05	0.29166\\
64.1	0.28787\\
64.15	0.28402\\
64.2	0.280110000000001\\
64.25	0.27613\\
64.3	0.2721\\
64.35	0.26802\\
64.4	0.26388\\
64.45	0.2597\\
64.5	0.255470000000001\\
64.55	0.251200000000001\\
64.6	0.2469\\
64.65	0.242550000000001\\
64.7	0.23817\\
64.75	0.23376\\
64.8	0.229320000000001\\
64.85	0.22485\\
64.9	0.220360000000001\\
64.95	0.215850000000001\\
65	0.211310000000001\\
65.05	0.206760000000001\\
65.1	0.20219\\
65.15	0.197610000000001\\
65.2	0.193020000000001\\
65.25	0.188420000000001\\
65.3	0.183810000000001\\
65.35	0.17919\\
65.4	0.174560000000001\\
65.45	0.16994\\
65.5	0.16531\\
65.55	0.160690000000001\\
65.6	0.15606\\
65.65	0.151440000000001\\
65.7	0.146830000000001\\
65.75	0.14222\\
65.8	0.13761\\
65.85	0.13302\\
65.9	0.128440000000001\\
65.95	0.12387\\
66	0.11931\\
66.05	0.11476\\
66.1	0.110230000000001\\
66.15	0.10572\\
66.2	0.101220000000001\\
66.25	0.0967400000000005\\
66.3	0.0922900000000002\\
66.35	0.0878500000000013\\
66.4	0.0834299999999999\\
66.45	0.0790300000000013\\
66.5	0.0746599999999997\\
66.55	0.070310000000001\\
66.6	0.0659799999999997\\
66.65	0.0616800000000008\\
66.7	0.0574000000000012\\
66.75	0.0531500000000005\\
66.8	0.0489300000000004\\
66.85	0.0447300000000013\\
66.9	0.040560000000001\\
66.95	0.0364199999999997\\
67	0.0323100000000007\\
67.05	0.0282300000000006\\
67.1	0.0241800000000012\\
67.15	0.0201600000000006\\
67.2	0.0161700000000007\\
67.25	0.0122100000000014\\
67.3	0.00829000000000057\\
67.35	0.00439000000000078\\
67.4	0.000530000000001252\\
67.45	-0.00329999999999941\\
67.5	-0.00709999999999944\\
67.55	-0.0108599999999992\\
67.6	-0.0145900000000001\\
67.65	-0.0182899999999986\\
67.7	-0.0219499999999986\\
67.75	-0.0255799999999997\\
67.8	-0.0291800000000002\\
67.85	-0.0327399999999987\\
67.9	-0.03627\\
67.95	-0.0397599999999994\\
68	-0.0432199999999998\\
68.05	-0.04664\\
68.1	-0.0500299999999996\\
68.15	-0.0533799999999989\\
68.2	-0.0566999999999993\\
68.25	-0.0599799999999995\\
68.3	-0.063229999999999\\
68.35	-0.0664400000000001\\
68.4	-0.0696199999999987\\
68.45	-0.0727599999999988\\
68.5	-0.0758700000000001\\
68.55	-0.0789399999999993\\
68.6	-0.0819799999999997\\
68.65	-0.0849799999999998\\
68.7	-0.0879499999999993\\
68.75	-0.0908800000000003\\
68.8	-0.0937799999999989\\
68.85	-0.0966499999999986\\
68.9	-0.0994700000000002\\
68.95	-0.102269999999999\\
69	-0.105029999999999\\
69.05	-0.107749999999999\\
69.1	-0.110439999999999\\
69.15	-0.113099999999999\\
69.2	-0.11572\\
69.25	-0.118309999999999\\
69.3	-0.120859999999999\\
69.35	-0.123379999999999\\
69.4	-0.125869999999999\\
69.45	-0.12832\\
69.5	-0.130739999999999\\
69.55	-0.13313\\
69.6	-0.135479999999999\\
69.65	-0.137799999999999\\
69.7	-0.14009\\
69.75	-0.142339999999999\\
69.8	-0.144559999999999\\
69.85	-0.146749999999999\\
69.9	-0.14891\\
69.95	-0.151039999999999\\
70	-0.153129999999999\\
70.05	-0.155189999999999\\
70.1	-0.15722\\
70.15	-0.159219999999999\\
70.2	-0.16119\\
70.25	-0.16313\\
70.3	-0.16503\\
70.35	-0.16691\\
70.4	-0.168749999999999\\
70.45	-0.17057\\
70.5	-0.17235\\
70.55	-0.17411\\
70.6	-0.175829999999999\\
70.65	-0.177529999999999\\
70.7	-0.1792\\
70.75	-0.180829999999999\\
70.8	-0.18244\\
70.85	-0.184019999999999\\
70.9	-0.185569999999999\\
70.95	-0.18709\\
71	-0.18859\\
71.05	-0.190059999999999\\
71.1	-0.191489999999999\\
71.15	-0.192909999999999\\
71.2	-0.19429\\
71.25	-0.195639999999999\\
71.3	-0.196969999999999\\
71.35	-0.19828\\
71.4	-0.199549999999999\\
71.45	-0.200799999999999\\
71.5	-0.202019999999999\\
71.55	-0.203219999999999\\
71.6	-0.204389999999999\\
71.65	-0.205539999999999\\
71.7	-0.206659999999999\\
71.75	-0.20775\\
71.8	-0.208819999999999\\
71.85	-0.20987\\
71.9	-0.21088\\
71.95	-0.21188\\
72	-0.21285\\
72.05	-0.213799999999999\\
72.1	-0.21472\\
72.15	-0.215619999999999\\
72.2	-0.216489999999999\\
72.25	-0.21735\\
72.3	-0.21817\\
72.35	-0.218979999999999\\
72.4	-0.219759999999999\\
72.45	-0.22052\\
72.5	-0.221259999999999\\
72.55	-0.221979999999999\\
72.6	-0.222669999999999\\
72.65	-0.223339999999999\\
72.7	-0.22399\\
72.75	-0.22462\\
72.8	-0.22523\\
72.85	-0.225809999999999\\
72.9	-0.22638\\
72.95	-0.22692\\
73	-0.227449999999999\\
73.05	-0.22795\\
73.1	-0.228429999999999\\
73.15	-0.22889\\
73.2	-0.22934\\
73.25	-0.22976\\
73.3	-0.23016\\
73.35	-0.230549999999999\\
73.4	-0.23091\\
73.45	-0.23126\\
73.5	-0.23159\\
73.55	-0.2319\\
73.6	-0.232189999999999\\
73.65	-0.23246\\
73.7	-0.232709999999999\\
73.75	-0.23295\\
73.8	-0.233169999999999\\
73.85	-0.23337\\
73.9	-0.233549999999999\\
73.95	-0.233719999999999\\
74	-0.23387\\
74.05	-0.233999999999999\\
74.1	-0.23412\\
74.15	-0.234209999999999\\
74.2	-0.234299999999999\\
74.25	-0.23436\\
74.3	-0.23441\\
74.35	-0.23445\\
74.4	-0.234469999999999\\
74.45	-0.234469999999999\\
74.5	-0.234459999999999\\
74.55	-0.23443\\
74.6	-0.234389999999999\\
74.65	-0.234329999999999\\
74.7	-0.234259999999999\\
74.75	-0.23417\\
74.8	-0.234069999999999\\
74.85	-0.233949999999999\\
74.9	-0.233829999999999\\
74.95	-0.23368\\
75	-0.23352\\
75.05	-0.23335\\
75.1	-0.233169999999999\\
75.15	-0.232969999999999\\
75.2	-0.23276\\
75.25	-0.232539999999999\\
75.3	-0.2323\\
75.35	-0.232049999999999\\
75.4	-0.231789999999999\\
75.45	-0.23152\\
75.5	-0.231229999999999\\
75.55	-0.23093\\
75.6	-0.230619999999999\\
75.65	-0.2303\\
75.7	-0.229959999999999\\
75.75	-0.22962\\
75.8	-0.229259999999999\\
75.85	-0.22889\\
75.9	-0.228509999999999\\
75.95	-0.22812\\
76	-0.22772\\
76.05	-0.227309999999999\\
76.1	-0.226889999999999\\
76.15	-0.226459999999999\\
76.2	-0.22601\\
76.25	-0.22556\\
76.3	-0.225099999999999\\
76.35	-0.22462\\
76.4	-0.224139999999999\\
76.45	-0.223649999999999\\
76.5	-0.22315\\
76.55	-0.222639999999999\\
76.6	-0.222119999999999\\
76.65	-0.221589999999999\\
76.7	-0.221049999999999\\
76.75	-0.220509999999999\\
76.8	-0.21995\\
76.85	-0.21939\\
76.9	-0.218819999999999\\
76.95	-0.21824\\
77	-0.217649999999999\\
77.05	-0.21705\\
77.1	-0.216449999999999\\
77.15	-0.215839999999999\\
77.2	-0.21522\\
77.25	-0.214589999999999\\
77.3	-0.213959999999999\\
77.35	-0.21332\\
77.4	-0.212669999999999\\
77.45	-0.212009999999999\\
77.5	-0.211349999999999\\
77.55	-0.210679999999999\\
77.6	-0.21001\\
77.65	-0.20933\\
77.7	-0.208639999999999\\
77.75	-0.207949999999999\\
77.8	-0.207249999999999\\
77.85	-0.20654\\
77.9	-0.20583\\
77.95	-0.205109999999999\\
78	-0.204389999999999\\
78.05	-0.203659999999999\\
78.1	-0.20292\\
78.15	-0.202179999999999\\
78.2	-0.20144\\
78.25	-0.20069\\
78.3	-0.199929999999999\\
78.35	-0.199179999999999\\
78.4	-0.198409999999999\\
78.45	-0.19764\\
78.5	-0.19687\\
78.55	-0.19609\\
78.6	-0.195309999999999\\
78.65	-0.19452\\
78.7	-0.19373\\
78.75	-0.192939999999999\\
78.8	-0.192139999999999\\
78.85	-0.191339999999999\\
78.9	-0.19053\\
78.95	-0.189719999999999\\
79	-0.188909999999999\\
79.05	-0.188099999999999\\
79.1	-0.187279999999999\\
79.15	-0.186459999999999\\
79.2	-0.18563\\
79.25	-0.184799999999999\\
79.3	-0.18397\\
79.35	-0.18314\\
79.4	-0.1823\\
79.45	-0.18146\\
79.5	-0.180619999999999\\
79.55	-0.179779999999999\\
79.6	-0.178929999999999\\
79.65	-0.178089999999999\\
79.7	-0.177239999999999\\
79.75	-0.17639\\
79.8	-0.175529999999999\\
79.85	-0.17468\\
79.9	-0.173819999999999\\
79.95	-0.17296\\
80	-0.172099999999999\\
};
\addplot [color=mycolor3,dotted,line width=1.5pt,forget plot]
  table[row sep=crcr]{%
50	-0.0267199999999992\\
50.05	-0.0267199999999992\\
50.1	-0.36408\\
50.15	-1.0416\\
50.2	-1.90258\\
50.25	-2.81764\\
50.3	-3.68674\\
50.35	-4.43639\\
50.4	-5.01657\\
50.45	-5.39889\\
50.5	-5.57619\\
50.55	-5.56236\\
50.6	-5.39082\\
50.65	-5.11032\\
50.7	-4.77765\\
50.75	-4.44852\\
50.8	-4.16904\\
50.85	-3.96983\\
50.9	-3.86419\\
50.95	-3.84964\\
51	-3.9118\\
51.05	-4.02896\\
51.1	-4.17644\\
51.15	-4.33022\\
51.2	-4.46969\\
51.25	-4.57941\\
51.3	-4.65003\\
51.35	-4.67845\\
51.4	-4.66732\\
51.45	-4.62397\\
51.5	-4.55883\\
51.55	-4.48362\\
51.6	-4.40963\\
51.65	-4.34623\\
51.7	-4.29988\\
51.75	-4.27377\\
51.8	-4.26791\\
51.85	-4.27969\\
51.9	-4.30469\\
51.95	-4.33751\\
52	-4.37267\\
52.05	-4.40528\\
52.1	-4.43158\\
52.15	-4.44918\\
52.2	-4.4572\\
52.25	-4.45613\\
52.3	-4.44754\\
52.35	-4.43374\\
52.4	-4.41736\\
52.45	-4.40096\\
52.5	-4.38671\\
52.55	-4.37615\\
52.6	-4.37012\\
52.65	-4.36876\\
52.7	-4.37161\\
52.75	-4.37778\\
52.8	-4.38607\\
52.85	-4.39525\\
52.9	-4.40416\\
52.95	-4.41185\\
53	-4.41773\\
53.05	-4.41043\\
53.1	-4.35823\\
53.15	-4.24219\\
53.2	-4.05842\\
53.25	-3.81583\\
53.3	-3.53256\\
53.35	-3.23162\\
53.4	-2.93629\\
53.45	-2.66621\\
53.5	-2.43493\\
53.55	-2.2489\\
53.6	-2.10805\\
53.65	-2.0072\\
53.7	-1.93791\\
53.75	-1.89038\\
53.8	-1.85503\\
53.85	-1.82361\\
53.9	-1.7899\\
53.95	-1.75002\\
54	-1.70231\\
54.05	-1.64699\\
54.1	-1.58566\\
54.15	-1.52075\\
54.2	-1.45494\\
54.25	-1.3907\\
54.3	-1.32999\\
54.35	-1.27408\\
54.4	-1.2235\\
54.45	-1.1781\\
54.5	-1.13721\\
54.55	-1.09986\\
54.6	-1.06489\\
54.65	-1.03125\\
54.7	-0.998209999999999\\
54.75	-0.965369999999999\\
54.8	-0.9326\\
54.85	-0.89999\\
54.9	-0.86774\\
54.95	-0.836119999999999\\
55	-0.805409999999999\\
55.05	-0.77619\\
55.1	-0.75139\\
55.15	-0.734909999999999\\
55.2	-0.729119999999999\\
55.25	-0.733829999999999\\
55.3	-0.74644\\
55.35	-0.76277\\
55.4	-0.77807\\
55.45	-0.78804\\
55.5	-0.789569999999999\\
55.55	-0.78114\\
55.6	-0.762849999999999\\
55.65	-0.736229999999999\\
55.7	-0.703779999999999\\
55.75	-0.6685\\
55.8	-0.63334\\
55.85	-0.60087\\
55.9	-0.572929999999999\\
55.95	-0.55057\\
56	-0.534059999999999\\
56.05	-0.522989999999999\\
56.1	-0.516459999999999\\
56.15	-0.51329\\
56.2	-0.51223\\
56.25	-0.512119999999999\\
56.3	-0.511979999999999\\
56.35	-0.51113\\
56.4	-0.50916\\
56.45	-0.505929999999999\\
56.5	-0.50153\\
56.55	-0.496199999999999\\
56.6	-0.490279999999999\\
56.65	-0.48412\\
56.7	-0.478039999999999\\
56.75	-0.472329999999999\\
56.8	-0.46716\\
56.85	-0.462639999999999\\
56.9	-0.45879\\
56.95	-0.455589999999999\\
57	-0.452939999999999\\
57.05	-0.45074\\
57.1	-0.448879999999999\\
57.15	-0.447249999999999\\
57.2	-0.445779999999999\\
57.25	-0.444389999999999\\
57.3	-0.443029999999999\\
57.35	-0.441649999999999\\
57.4	-0.440219999999999\\
57.45	-0.43866\\
57.5	-0.43694\\
57.55	-0.435009999999999\\
57.6	-0.432849999999999\\
57.65	-0.43047\\
57.7	-0.427879999999999\\
57.75	-0.42514\\
57.8	-0.422299999999999\\
57.85	-0.419409999999999\\
57.9	-0.416549999999999\\
57.95	-0.413759999999999\\
58	-0.411079999999999\\
58.05	-0.408549999999999\\
58.1	-0.40616\\
58.15	-0.403929999999999\\
58.2	-0.401829999999999\\
58.25	-0.39985\\
58.3	-0.39795\\
58.35	-0.39612\\
58.4	-0.394329999999999\\
58.45	-0.39256\\
58.5	-0.390789999999999\\
58.55	-0.389019999999999\\
58.6	-0.38723\\
58.65	-0.385439999999999\\
58.7	-0.38364\\
58.75	-0.381849999999999\\
58.8	-0.380059999999999\\
58.85	-0.37829\\
58.9	-0.376549999999999\\
58.95	-0.374829999999999\\
59	-0.373139999999999\\
59.05	-0.371479999999999\\
59.1	-0.369859999999999\\
59.15	-0.36827\\
59.2	-0.366709999999999\\
59.25	-0.365169999999999\\
59.3	-0.363659999999999\\
59.35	-0.36218\\
59.4	-0.360709999999999\\
59.45	-0.35927\\
59.5	-0.357839999999999\\
59.55	-0.356439999999999\\
59.6	-0.355049999999999\\
59.65	-0.353689999999999\\
59.7	-0.352349999999999\\
59.75	-0.35103\\
59.8	-0.349729999999999\\
59.85	-0.348459999999999\\
59.9	-0.347219999999999\\
59.95	-0.345999999999999\\
60	-0.34481\\
60.05	-0.34364\\
60.1	-0.34251\\
60.15	-0.341399999999999\\
60.2	-0.340319999999999\\
60.25	-0.339269999999999\\
60.3	-0.338249999999999\\
60.35	-0.33726\\
60.4	-0.336289999999999\\
60.45	-0.33536\\
60.5	-0.334449999999999\\
60.55	-0.333569999999999\\
60.6	-0.332719999999999\\
60.65	-0.331899999999999\\
60.7	-0.33111\\
60.75	-0.330349999999999\\
60.8	-0.32961\\
60.85	-0.32891\\
60.9	-0.32823\\
60.95	-0.32759\\
61	-0.326969999999999\\
61.05	-0.326379999999999\\
61.1	-0.325819999999999\\
61.15	-0.32529\\
61.2	-0.324789999999999\\
61.25	-0.324319999999999\\
61.3	-0.323879999999999\\
61.35	-0.32346\\
61.4	-0.323079999999999\\
61.45	-0.322729999999999\\
61.5	-0.322399999999999\\
61.55	-0.322109999999999\\
61.6	-0.321839999999999\\
61.65	-0.321599999999999\\
61.7	-0.321389999999999\\
61.75	-0.32121\\
61.8	-0.321059999999999\\
61.85	-0.32093\\
61.9	-0.32084\\
61.95	-0.32077\\
62	-0.320729999999999\\
62.05	-0.32072\\
62.1	-0.320729999999999\\
62.15	-0.320779999999999\\
62.2	-0.320849999999999\\
62.25	-0.32095\\
62.3	-0.32107\\
62.35	-0.321229999999999\\
62.4	-0.321409999999999\\
62.45	-0.32161\\
62.5	-0.321839999999999\\
62.55	-0.3221\\
62.6	-0.32239\\
62.65	-0.322699999999999\\
62.7	-0.323029999999999\\
62.75	-0.323399999999999\\
62.8	-0.323779999999999\\
62.85	-0.32419\\
62.9	-0.324629999999999\\
62.95	-0.325089999999999\\
63	-0.325569999999999\\
63.05	-0.326079999999999\\
63.1	-0.32661\\
63.15	-0.32717\\
63.2	-0.327749999999999\\
63.25	-0.328349999999999\\
63.3	-0.328969999999999\\
63.35	-0.32961\\
63.4	-0.330279999999999\\
63.45	-0.33097\\
63.5	-0.33168\\
63.55	-0.332409999999999\\
63.6	-0.333159999999999\\
63.65	-0.33393\\
63.7	-0.334719999999999\\
63.75	-0.335539999999999\\
63.8	-0.33637\\
63.85	-0.337219999999999\\
63.9	-0.338089999999999\\
63.95	-0.33897\\
64	-0.339879999999999\\
64.05	-0.3408\\
64.1	-0.34174\\
64.15	-0.3427\\
64.2	-0.343669999999999\\
64.25	-0.344659999999999\\
64.3	-0.34566\\
64.35	-0.34669\\
64.4	-0.34772\\
64.45	-0.348769999999999\\
64.5	-0.349839999999999\\
64.55	-0.350919999999999\\
64.6	-0.352009999999999\\
64.65	-0.35312\\
64.7	-0.35424\\
64.75	-0.35537\\
64.8	-0.35652\\
64.85	-0.357679999999999\\
64.9	-0.358849999999999\\
64.95	-0.360029999999999\\
65	-0.361219999999999\\
65.05	-0.362419999999999\\
65.1	-0.36363\\
65.15	-0.36485\\
65.2	-0.366079999999999\\
65.25	-0.367329999999999\\
65.3	-0.368569999999999\\
65.35	-0.369829999999999\\
65.4	-0.371099999999999\\
65.45	-0.372369999999999\\
65.5	-0.37365\\
65.55	-0.37494\\
65.6	-0.37623\\
65.65	-0.377529999999999\\
65.7	-0.37883\\
65.75	-0.380139999999999\\
65.8	-0.38146\\
65.85	-0.382779999999999\\
65.9	-0.384099999999999\\
65.95	-0.385429999999999\\
66	-0.38676\\
66.05	-0.3881\\
66.1	-0.38944\\
66.15	-0.390779999999999\\
66.2	-0.392119999999999\\
66.25	-0.39347\\
66.3	-0.39481\\
66.35	-0.396159999999999\\
66.4	-0.39751\\
66.45	-0.398859999999999\\
66.5	-0.40021\\
66.55	-0.401559999999999\\
66.6	-0.402909999999999\\
66.65	-0.40426\\
66.7	-0.405609999999999\\
66.75	-0.406949999999999\\
66.8	-0.4083\\
66.85	-0.40964\\
66.9	-0.410979999999999\\
66.95	-0.412319999999999\\
67	-0.41365\\
67.05	-0.414979999999999\\
67.1	-0.416309999999999\\
67.15	-0.41764\\
67.2	-0.418959999999999\\
67.25	-0.420269999999999\\
67.3	-0.421589999999999\\
67.35	-0.42289\\
67.4	-0.424199999999999\\
67.45	-0.425489999999999\\
67.5	-0.426779999999999\\
67.55	-0.428069999999999\\
67.6	-0.429349999999999\\
67.65	-0.430619999999999\\
67.7	-0.431889999999999\\
67.75	-0.433149999999999\\
67.8	-0.434399999999999\\
67.85	-0.435639999999999\\
67.9	-0.436879999999999\\
67.95	-0.438109999999999\\
68	-0.439329999999999\\
68.05	-0.440539999999999\\
68.1	-0.44175\\
68.15	-0.44295\\
68.2	-0.444129999999999\\
68.25	-0.445309999999999\\
68.3	-0.446479999999999\\
68.35	-0.44764\\
68.4	-0.44879\\
68.45	-0.44994\\
68.5	-0.45107\\
68.55	-0.45219\\
68.6	-0.4533\\
68.65	-0.4544\\
68.7	-0.455489999999999\\
68.75	-0.456569999999999\\
68.8	-0.45764\\
68.85	-0.458699999999999\\
68.9	-0.459739999999999\\
68.95	-0.46078\\
69	-0.461799999999999\\
69.05	-0.46282\\
69.1	-0.463819999999999\\
69.15	-0.464809999999999\\
69.2	-0.46578\\
69.25	-0.466749999999999\\
69.3	-0.4677\\
69.35	-0.46864\\
69.4	-0.469569999999999\\
69.45	-0.47049\\
69.5	-0.471399999999999\\
69.55	-0.472289999999999\\
69.6	-0.47317\\
69.65	-0.474029999999999\\
69.7	-0.474889999999999\\
69.75	-0.47573\\
69.8	-0.476559999999999\\
69.85	-0.47737\\
69.9	-0.478179999999999\\
69.95	-0.478969999999999\\
70	-0.479749999999999\\
70.05	-0.48051\\
70.1	-0.48126\\
70.15	-0.481999999999999\\
70.2	-0.48272\\
70.25	-0.483429999999999\\
70.3	-0.48413\\
70.35	-0.484819999999999\\
70.4	-0.48549\\
70.45	-0.486149999999999\\
70.5	-0.486789999999999\\
70.55	-0.48743\\
70.6	-0.488049999999999\\
70.65	-0.48865\\
70.7	-0.48924\\
70.75	-0.489819999999999\\
70.8	-0.49039\\
70.85	-0.490939999999999\\
70.9	-0.491479999999999\\
70.95	-0.49201\\
71	-0.49252\\
71.05	-0.49302\\
71.1	-0.49351\\
71.15	-0.49398\\
71.2	-0.494439999999999\\
71.25	-0.49489\\
71.3	-0.495329999999999\\
71.35	-0.495749999999999\\
71.4	-0.49616\\
71.45	-0.49656\\
71.5	-0.496939999999999\\
71.55	-0.49731\\
71.6	-0.497669999999999\\
71.65	-0.498019999999999\\
71.7	-0.498349999999999\\
71.75	-0.49867\\
71.8	-0.49898\\
71.85	-0.49928\\
71.9	-0.49956\\
71.95	-0.499829999999999\\
72	-0.500089999999999\\
72.05	-0.50034\\
72.1	-0.500579999999999\\
72.15	-0.500799999999999\\
72.2	-0.501009999999999\\
72.25	-0.501209999999999\\
72.3	-0.501399999999999\\
72.35	-0.50158\\
72.4	-0.50174\\
72.45	-0.501899999999999\\
72.5	-0.502039999999999\\
72.55	-0.502179999999999\\
72.6	-0.502299999999999\\
72.65	-0.502409999999999\\
72.7	-0.502509999999999\\
72.75	-0.50259\\
72.8	-0.502669999999999\\
72.85	-0.502739999999999\\
72.9	-0.5028\\
72.95	-0.502839999999999\\
73	-0.502879999999999\\
73.05	-0.502899999999999\\
73.1	-0.50292\\
73.15	-0.502929999999999\\
73.2	-0.50292\\
73.25	-0.502909999999999\\
73.3	-0.502879999999999\\
73.35	-0.50285\\
73.4	-0.502809999999999\\
73.45	-0.502759999999999\\
73.5	-0.502699999999999\\
73.55	-0.502619999999999\\
73.6	-0.502549999999999\\
73.65	-0.502459999999999\\
73.7	-0.502359999999999\\
73.75	-0.50226\\
73.8	-0.50214\\
73.85	-0.502019999999999\\
73.9	-0.50189\\
73.95	-0.501749999999999\\
74	-0.5016\\
74.05	-0.501449999999999\\
74.1	-0.50128\\
74.15	-0.50111\\
74.2	-0.500939999999999\\
74.25	-0.500749999999999\\
74.3	-0.500559999999999\\
74.35	-0.50036\\
74.4	-0.50015\\
74.45	-0.49994\\
74.5	-0.499709999999999\\
74.55	-0.49949\\
74.6	-0.499249999999999\\
74.65	-0.499009999999999\\
74.7	-0.49876\\
74.75	-0.49851\\
74.8	-0.49825\\
74.85	-0.497979999999999\\
74.9	-0.49771\\
74.95	-0.49743\\
75	-0.497139999999999\\
75.05	-0.496849999999999\\
75.1	-0.49656\\
75.15	-0.496259999999999\\
75.2	-0.49595\\
75.25	-0.49564\\
75.3	-0.49532\\
75.35	-0.494999999999999\\
75.4	-0.494669999999999\\
75.45	-0.494339999999999\\
75.5	-0.494009999999999\\
75.55	-0.49367\\
75.6	-0.49332\\
75.65	-0.49297\\
75.7	-0.49262\\
75.75	-0.492259999999999\\
75.8	-0.491899999999999\\
75.85	-0.491529999999999\\
75.9	-0.49116\\
75.95	-0.49079\\
76	-0.49041\\
76.05	-0.490029999999999\\
76.1	-0.489649999999999\\
76.15	-0.48926\\
76.2	-0.488869999999999\\
76.25	-0.48847\\
76.3	-0.488079999999999\\
76.35	-0.487679999999999\\
76.4	-0.48727\\
76.45	-0.48687\\
76.5	-0.486459999999999\\
76.55	-0.48605\\
76.6	-0.485639999999999\\
76.65	-0.485219999999999\\
76.7	-0.484799999999999\\
76.75	-0.48438\\
76.8	-0.48396\\
76.85	-0.48354\\
76.9	-0.483109999999999\\
76.95	-0.482679999999999\\
77	-0.48225\\
77.05	-0.481819999999999\\
77.1	-0.481389999999999\\
77.15	-0.480949999999999\\
77.2	-0.480519999999999\\
77.25	-0.480079999999999\\
77.3	-0.47964\\
77.35	-0.4792\\
77.4	-0.478759999999999\\
77.45	-0.478319999999999\\
77.5	-0.477869999999999\\
77.55	-0.477429999999999\\
77.6	-0.47699\\
77.65	-0.476539999999999\\
77.7	-0.476089999999999\\
77.75	-0.475649999999999\\
77.8	-0.475199999999999\\
77.85	-0.474749999999999\\
77.9	-0.474299999999999\\
77.95	-0.47385\\
78	-0.4734\\
78.05	-0.472949999999999\\
78.1	-0.472499999999999\\
78.15	-0.472049999999999\\
78.2	-0.4716\\
78.25	-0.47115\\
78.3	-0.470699999999999\\
78.35	-0.470249999999999\\
78.4	-0.469799999999999\\
78.45	-0.469349999999999\\
78.5	-0.4689\\
78.55	-0.46845\\
78.6	-0.46801\\
78.65	-0.46756\\
78.7	-0.467109999999999\\
78.75	-0.466659999999999\\
78.8	-0.46622\\
78.85	-0.465769999999999\\
78.9	-0.465319999999999\\
78.95	-0.464879999999999\\
79	-0.46444\\
79.05	-0.463989999999999\\
79.1	-0.46355\\
79.15	-0.463109999999999\\
79.2	-0.462669999999999\\
79.25	-0.462229999999999\\
79.3	-0.46179\\
79.35	-0.461349999999999\\
79.4	-0.46092\\
79.45	-0.46048\\
79.5	-0.46005\\
79.55	-0.45961\\
79.6	-0.459179999999999\\
79.65	-0.458749999999999\\
79.7	-0.45832\\
79.75	-0.457889999999999\\
79.8	-0.45747\\
79.85	-0.457039999999999\\
79.9	-0.456619999999999\\
79.95	-0.456189999999999\\
80	-0.455769999999999\\
};
\end{axis}
\end{tikzpicture}%

  }
  \caption[Zoomed view of surge distance time response of serial system.]{Zoomed view of surge distance time response given in \fig{res:serial-timeresp}.}
  \label{fig:res:serial-sd-zoom}
\end{figure}

In contrast to the parallel case, differences in the controller response and performance can be observed for each of the control approaches.
The cooperative controller response is qualitatively similar to that of the centralized controller.
It is, however, somewhat less aggressive in its regulation of the output pressure, evidenced by its less extreme change from high to low torque in the downstream compressor and faster return to the steady-state torque value, as shown in \fig{res:serial-timeresp}.
As a result, the output pressure of the downstream compressor has a much lower overshoot than either the centralized or non-cooperative controller.

Similarly, in the upstream compressor, the cooperative controller's torque input is smoother than that of the centralized controller with no uptick near \u{70}{s}, and a faster convergence of both inputs to their steady-state values.
Accordingly, the upstream compressor's output pressure also converges faster for the cooperative controller than for the centralized one.
The surge distances of the centralized and cooperative controllers, however, show an almost identical response for the downstream compressor and a similar response upstream, though that of the cooperative controller has a greater increase near \u{58}{s} (see \fig{res:serial-sd-zoom}).

The non-cooperative controller response has a much different characteristic than that of the centralized controller.
There is no significant initial increase in the torque input to either compressor when the disturbance is applied; as a result, both compressors are pushed further towards surge than in the centralized or cooperative case, and the surge distances are also slower to converge.
The maximum disturbance to the output pressures of both compressors for the non-cooperative controller is accordingly reduced by approximately 40\% when compared to the centralized case.

The integral squared error (ISE) and integral absolute error (IAE) are shown in \tab{res:performance:ser-ise} for all controllers.

\begin{table}
  \centering
  \caption{Integral squared error (ISE) and integral absolute error (IAE) measures for serial controllers.}
  \begin{tabular}{ccccccc}
    \toprule
    & \multicolumn{2}{c}{Centralized} & \multicolumn{2}{c}{Cooperative} & \multicolumn{2}{c}{Non-cooperative}\\
    & ISE & IAE & ISE & IAE &ISE & IAE \\
    \midrule
    \gi{torque} &   0.0012 &    0.012 &    0.001 &   0.0079 &   0.0015 &    0.019\\
    \gi{ur} &  8.6e-05 &   0.0037 &  5.6e-05 &   0.0024 &  0.00011 &   0.0053\\
    \gi{pd} &  0.00096 &    0.011 &   0.0006 &   0.0067 &  0.00081 &    0.014\\
    \gi{sd} &    0.052 &    0.063 &    0.031 &    0.041 &    0.078 &     0.13\\
    \gii{torque} &   0.0027 &    0.016 &  0.00095 &   0.0073 &   0.0017 &    0.021\\
    \gii{ur} &  1.1e-05 &   0.0011 &  2.7e-05 &   0.0016 &  3.2e-05 &   0.0026\\
    \gii{pd} &  0.00047 &   0.0051 &  0.00058 &   0.0049 &  0.00029 &    0.005\\
    \gii{sd} &    0.056 &    0.032 &    0.064 &    0.035 &      0.1 &    0.064\\
    \bottomrule
  \end{tabular}
  \label{tab:res:performance:ser-ise}
\end{table}



\subsection{Computational Efficiency}

The computational cost of the centralized and distributed control approaches was evaluated and is presented in \fig{results:compcost}.%
\footnote{Tests performed using a \cpp{} implementation and run on a Intel\textregistered{} Core\texttrademark{} i5-540M \u{2.53}{\giga\hertz} processor.} 
The computation times for the distributed controllers assume a parallelized implementation where each sub-controller is executed using a separate processor. 

\begin{figure}
  \resizebox{\linewidth}{!}{%
    % This file was created by matlab2tikz.
%
\definecolor{mycolor1}{rgb}{0.00000,0.44700,0.74100}%
\definecolor{mycolor2}{rgb}{0.85000,0.32500,0.09800}%
\definecolor{mycolor3}{rgb}{0.92900,0.69400,0.12500}%
%
\begin{tikzpicture}

\begin{axis}[%
width=0.38\linewidth,
height=0.3\linewidth,
at={(0\linewidth,0\linewidth)},
scale only axis,
xmin=1,
xmax=9,
xlabel={Number of solver iterations},
xmajorgrids,
ymin=0.2,
ymax=0.55,
ylabel={Computation time [ms]},
ymajorgrids,
axis background/.style={fill=white},
title style={font=\bfseries,yshift=2.1ex},
title={Parallel System},
ylabel near ticks,
legend columns=-1,
legend style={at={(0.03,1.25)},anchor=north west,legend cell align=left,align=left,draw=none}
]
\addplot [color=mycolor1,solid,line width=1pt,mark size=3.0pt,mark=o,mark options={solid}]
  table[row sep=crcr]{%
1	0.3476\\
2	0.3476\\
3	0.3476\\
4	0.3476\\
5	0.3476\\
6	0.3476\\
7	0.3476\\
8	0.3476\\
9	0.3476\\
};
\addlegendentry{Centralized};

\addplot [color=mycolor2,solid,line width=1pt,mark size=2.5pt,mark=square,mark options={solid}]
  table[row sep=crcr]{%
1 0.3636\\
2 0.3760\\
3 0.3845\\
4 0.4085\\
5 0.4130\\
6 0.4191\\
7 0.4384\\
8 0.4381\\
9 0.4502\\
};
\addlegendentry{Cooperative};

\addplot [color=mycolor3,solid,line width=1pt,mark size=3.0pt,mark=x,mark options={solid}]
  table[row sep=crcr]{%
1 0.2302\\
2 0.2399\\
3 0.2496\\
4 0.2622\\
5 0.2691\\
6 0.2806\\
7 0.2900\\
8 0.3055\\
9 0.3137\\
};
\addlegendentry{Non-cooperative};

\end{axis}

\begin{axis}[%
% width=0.761\linewidth,
% height=0.597\linewidth,
width=0.38\linewidth,
height=0.3\linewidth,
at={(0.5\linewidth,0\linewidth)},
scale only axis,
xmin=1,
xmax=9,
xlabel={Number of solver iterations},
xmajorgrids,
ymin=0.2,
ymax=0.55,
% ylabel={Average computation time [ms]},
ymajorgrids,
axis background/.style={fill=white},
title style={font=\bfseries,yshift=2.1ex},
title={Serial System}
]
\addplot [color=mycolor1,solid,line width=1pt,mark size=3.0pt,mark=o,mark options={solid},forget plot]
  table[row sep=crcr]{%
1	0.3713\\
2	0.3713\\
3	0.3713\\
4	0.3713\\
5	0.3713\\
6	0.3713\\
7	0.3713\\
8	0.3713\\
9	0.3713\\
};
\addplot [color=mycolor2,solid,line width=1pt,mark size=2.5pt,mark=square,mark options={solid},forget plot]
  table[row sep=crcr]{%
1 0.3437\\
2 0.3620\\
3 0.3670\\
4 0.3794\\
5 0.3895\\
6 0.4135\\
7 0.4199\\
8 0.4249\\
9 0.4428\\
};
\addplot [color=mycolor3,solid,line width=1pt,mark size=3.0pt,mark=x,mark options={solid},forget plot]
  table[row sep=crcr]{%
1 0.2169\\
2 0.2224\\
3 0.2294\\
4 0.2415\\
5 0.2483\\
6 0.2576\\
7 0.2694\\
8 0.2773\\
9 0.2887\\
};
\end{axis}

\end{tikzpicture}%

  }
  \caption[Controller computation time per iteration.]{Average controller computation time per iteration for centralized and distributed controllers, as a function of the number of solver iterations used. Controller performance gains for solver iterations greater than 3 are marginal. Disturbance (described above) is applied at \u{50}{s} and total simulation time is \u{500}{s}.}
  \label{fig:results:compcost}
\end{figure}


As expected, the non-cooperative controller achieves the lowest computation times for both the parallel and the serial system.
In both cases, the cooperative controller has a computation time approximately on par with the centralized controller, though slightly lower for few solver iterations.
The computational cost of the cooperative controller relative to the centralized controller is dependent on the operating point: for the parallel system, for example, at the non-disturbed, initial state, a single solver iteration of the cooperative controller requires as much computation time as the centralized controller, but the cooperative controller executes faster at the disturbed operating point.
The number of solver iterations for which the cooperative approach requires more computation time than the centralized approach is thus shifted depending on the simulation case considered.

It should be noted that increasing the number of solver iterations above 3 had no measurable effect on performance.

The significant advantage in computation time demonstrated by the non-cooperative controller is a result of the reduced size of its prediction matrices, which are multiplied as described in \sect{mpc} to generate the \g{qp}.
The computational cost of the \g{qp} generation is approximately linear in the number of outputs used, and the non-cooperative approach considers fewer outputs than the centralized or cooperative approaches (see \tab{res:parallel-weights}).

As described in~\cite{jones2016}, for the systems considered here, the QP-generation step has a much higher computational cost than the QP-solving step.
% The percentage of computation time spent on solving QPs (including the time required to update the linear QP term after each solver iteration for the distributed controllers) is shown in \fig{res:qp-compcost}.
% The remaining time is largely dedicated to QP generation.
% These results are implementation dependent and the computation time required for QP generation could be reduced by using a more efficient linear algebra library that that provided in \eigen{}, however it would still contribute significantly to the total computational cost.
%
% Using 2 or 3 solver iterations, less than 10\% of the total computation time is spent solving QPs for all control approaches. 
There is thus a limited scope for decreasing the required computation time without decreasing the number of states used to generate the QP problem.
% There is thus a limited scope for decreasing the required computation time without decreasing either the number of outputs or states used to generate the QP problem.
% Furthermore, the number of outputs used for the cooperative QP is fixed and cannot be decreased, since the sub-controllers have the same cost function, and for the non-cooperative controller only 2 outputs are used so it cannot be further decreased.
% Further gains in computational efficiency would therefore require a reduced number of states to be considered when generating the QP problem.



\section{Conclusion}
% \input{conclusion.tex}

\bibliography{end/MasterThesis.bib}
\end{document}
