In order to test the controller performance of the various control implementations, a single benchmark disturbance case was chosen.
It is designed to mimic a typical disturbance downstream of the system such as the shutdown of another compressor.
The disturbances are based on the worst-case disturbance for a single compressor discussed in \cite{Cortinovis2015}, and are considered the worst-case for the multi-compressor system as well.
The disturbances used were as follows:

\begin{itemize}
  \item \textbf{parallel}: common tank discharge valve (\g{ut}) 70\% -> 40\% open;
  \item \textbf{serial}: downstream compressor discharge valve (\gii{ud}) 39\% -> 29\% open.
\end{itemize}
In the following plots, the disturbances are persistent and applied at \u{50}{s}, after the system has reached steady state.

The distributed controllers in this section were implemented using 3 controller iterations. 
As discussed in \cite{Jones2016}, this number was determined sufficient for convergence of the applied inputs.

\subsection{Parallel System Control Performance}
The time response of each controller for the parallel system is shown in \fig{res:parallel-timeresp}.
The responses obtained using each of the three controllers are virtually identical -- in this case, using a distributed control approach has no performance penalty and a potential decrease in computational cost (see below).
The main reason for this result is the very low degree of coupling between the two parallel compressors.


\begin{figure}
  {\centering\small\textbf{Parallel System}\\Both Compressors\\[0.5em]}
  \resizebox{\linewidth}{!}{%
    % This file was created by matlab2tikz.
%
\definecolor{mycolor1}{rgb}{0.00000,0.44700,0.74100}%
\definecolor{mycolor2}{rgb}{0.85000,0.32500,0.09800}%
\definecolor{mycolor3}{rgb}{0.92900,0.69400,0.12500}%
%
\begin{tikzpicture}

\begin{axis}[%
width=5cm,
height=4cm,
at={(0\linewidth,0\linewidth)},
scale only axis,
xmin=0,
xmax=500,
xlabel={Time [s]},
xmajorgrids,
ymin=1.08,
ymax=1.28,
ylabel={Tank Output Pressure [atm]},
% ylabel near ticks,
ylabel style={yshift=-0.1cm},
ymajorgrids,
axis background/.style={fill=white},
title style={font=\bfseries},
% title={Tank Output Pressure},
axis x line*=bottom,
axis y line*=left,
legend style={legend cell align=left,align=left,draw=white!15!black}
]
\input{figures/parallel_p.tex}
\end{axis}


\begin{axis}[%
width=5cm,
height=4cm,
at={(7cm,0\linewidth)},
scale only axis,
xmin=0,
xmax=500,
xlabel={Time [s]},
xmajorgrids,
ymin=-2,
ymax=1,
ylabel={Relative Surge Control Distance [\%]},
% ylabel near ticks,
ylabel style={yshift=-0.1cm},
ymajorgrids,
axis background/.style={fill=white},
title style={font=\bfseries},
% title={Surge Distance},
axis x line*=bottom,
axis y line*=left
]
\input{figures/parallel_sd.tex}
\end{axis}

\begin{axis}[%
width=5cm,
height=4cm,
at={(0cm,6cm)},
scale only axis,
xmin=0,
xmax=500,
xlabel={Time [s]},
xmajorgrids,
ymin=0.2,
ymax=0.38,
ylabel={Normalized Torque Setting},
% ylabel near ticks,
ylabel style={yshift=-0.1cm},
ymajorgrids,
axis background/.style={fill=white},
title style={font=\bfseries},
% title={Normalized Torque Setting},
axis x line*=bottom,
axis y line*=left
]
\input{figures/parallel_td.tex}
\end{axis}

\begin{axis}[%
width=5cm,
height=4cm,
at={(7cm, 6cm)},
scale only axis,
xmin=0,
xmax=500,
xlabel={Time [s]},
xmajorgrids,
ymin=0,
ymax=0.07,
ylabel={Normalized Recycle Valve Opening},
% ylabel style={font=\tiny},
% ylabel near ticks,
ylabel style={yshift=-0.1cm},
ymajorgrids,
axis background/.style={fill=white},
% title style={font=\tiny\bfseries},
% title={Normalized Recycle Valve Opening},
axis x line*=bottom,
axis y line*=left
]
\input{figures/parallel_ur.tex}
\end{axis}
\end{tikzpicture}

  }
  \caption[Time response of parallel system.]{Comparison of parallel system time responses of centralized and distributed controllers using 3 solver iterations. Output disturbance is applied at \u{50}{s}.}
  \label{fig:res:parallel-timeresp}
\end{figure}

\begin{figure}
  \centering
  \resizebox{0.5\linewidth}{!}{%
    \input{figures/parallel_sd_zoom.tex}
  }
  \caption[Zoomed view of surge distance time response of parallel system.]{Zoomed view of parallel system surge distance time response given in \fig{res:parallel-timeresp}.}
  \label{fig:res:parallel-sd-zoom}
\end{figure}

The integral squared error (ISE) and integral absolute error (IAE) are shown in \tab{res:performance:ser-ise} for all controllers.

\begin{table}
  \centering
  \caption{Integral squared error (ISE) and integral absolute error (IAE) measures for parallel controllers.}
  \begin{tabular}{ccccccc}
    \toprule
    & \multicolumn{2}{c}{Centralized} & \multicolumn{2}{c}{Cooperative} & \multicolumn{2}{c}{Non-cooperative}\\
    \midrule
    & ISE & IAE & ISE & IAE &ISE & IAE \\
    \g{torque} & 0.0026 &    0.027 &   0.0026 &    0.028 &   0.0026 &    0.028 \\
    \g{ur} &  0.00012 &   0.0057 &  0.00011 &   0.0056 &  0.00012 &   0.0059 \\
    \g{sd} &    0.035 &    0.057 &    0.033 &    0.054 &    0.036 &    0.059 \\
    \g{pt} &   0.0023 &    0.025 &   0.0024 &    0.026 &   0.0023 &    0.025 \\
    \bottomrule
  \end{tabular}
  \label{tab:res:performance:par-ise}
\end{table}



\subsection{Serial System Control Performance}

The time response of the serial system is shown in \fig{res:serial-timeresp}.
A zoomed view of the initial surge distance response is shown in \fig{res:serial-sd-zoom}.

\begin{figure}
  {\centering\small\textbf{Serial System}\\Upstream Compressor\\[0.5em]}
  \resizebox{\linewidth}{!}{%
    \definecolor{mycolor1}{rgb}{0.00000,0.44700,0.74100}%
\definecolor{mycolor2}{rgb}{0.85000,0.32500,0.09800}%
\definecolor{mycolor3}{rgb}{0.92900,0.69400,0.12500}%
%
\begin{tikzpicture}

\begin{axis}[%
width=5cm,
height=4cm,
at={(0\linewidth,0\linewidth)},
scale only axis,
xmin=0,
xmax=320,
xlabel={Time [s]},
xmajorgrids,
ymin=1,
ymax=1.18,
ylabel={Tank Output Pressure [atm]},
ylabel style={yshift=-0.1cm},
ymajorgrids,
axis background/.style={fill=white},
% title style={font=\bfseries},
% title={Tank Output Pressure},
axis x line*=bottom,
axis y line*=left,
legend style={legend cell align=left,align=left,draw=white!15!black}
]
\input{figures/serial_p1.tex}
\end{axis}

\begin{axis}[%
width=5cm,
height=4cm,
at={(7cm,0cm)},
scale only axis,
xmin=0,
xmax=320,
xlabel={Time [s]},
xmajorgrids,
ymin=-3.5,
ymax=1,
ylabel={Relative Surge Control Distance [\%]},
ylabel style={yshift=-0.1cm},
ymajorgrids,
axis background/.style={fill=white},
% title style={font=\bfseries},
% title={Surge Control Distance},
axis x line*=bottom,
axis y line*=left
]
\input{figures/serial_sd1.tex}
\end{axis}


\begin{axis}[%
width=5cm,
height=4cm,
at={(0cm,6cm)},
scale only axis,
xmin=0,
xmax=320,
xlabel={Time [s]},
xmajorgrids,
ymin=0.15,
ymax=0.45,
ylabel={Normalized Torque Setting},
ylabel style={yshift=-0.1cm},
ymajorgrids,
axis background/.style={fill=white},
% title style={font=\bfseries},
% title={Normalized Torque Setting},
axis x line*=bottom,
axis y line*=left
]
\input{figures/serial_td1.tex}
\end{axis}

\begin{axis}[%
width=5cm,
height=4cm,
at={(7cm,6cm)},
scale only axis,
xmin=0,
xmax=320,
xlabel={Time [s]},
xmajorgrids,
ymin=0,
ymax=0.08,
ylabel={Normalized Recycle Valve Opening},
ylabel style={yshift=-0.1cm},
ymajorgrids,
axis background/.style={fill=white},
% title style={font=\bfseries},
% title={Normalized Recycle Valve Opening},
axis x line*=bottom,
axis y line*=left
]
\input{figures/serial_ur1.tex}
\end{axis}
\end{tikzpicture}

  }

  {\centering\small\textbf{Serial System}\\Downstream Compressor\\[0.5em]}
  \resizebox{\linewidth}{!}{%
    \definecolor{mycolor1}{rgb}{0.00000,0.44700,0.74100}%
\definecolor{mycolor2}{rgb}{0.85000,0.32500,0.09800}%
\definecolor{mycolor3}{rgb}{0.92900,0.69400,0.12500}%
%
\begin{tikzpicture}

\begin{axis}[%
width=5cm,
height=4cm,
at={(0\linewidth,0\linewidth)},
scale only axis,
xmin=0,
xmax=320,
xlabel={Time [s]},
xmajorgrids,
ymin=1.15,
ymax=1.4,
ylabel={Tank Output Pressure [atm]},
ylabel style={yshift=-0.1cm},
ymajorgrids,
axis background/.style={fill=white},
% title style={font=\bfseries},
% title={Tank Output Pressure},
axis x line*=bottom,
axis y line*=left,
legend style={legend cell align=left,align=left,draw=white!15!black}
]
\input{figures/serial_p2.tex}
\end{axis}

\begin{axis}[%
width=5cm,
height=4cm,
at={(7cm,0cm)},
scale only axis,
xmin=0,
xmax=320,
xlabel={Time [s]},
xmajorgrids,
ymin=-6,
ymax=1,
ylabel={Relative Surge Control Distance [\%]},
ylabel style={yshift=-0.1cm},
ymajorgrids,
axis background/.style={fill=white},
% title style={font=\bfseries},
% title={Surge Control Distance},
axis x line*=bottom,
axis y line*=left
]
\input{figures/serial_sd2.tex}
\end{axis}


\begin{axis}[%
width=5cm,
height=4cm,
at={(0cm,6cm)},
scale only axis,
xmin=0,
xmax=320,
xlabel={Time [s]},
xmajorgrids,
ymin=0,
ymax=0.7,
ylabel={Normalized Torque Setting},
ylabel style={yshift=-0.1cm},
ymajorgrids,
axis background/.style={fill=white},
% title style={font=\bfseries},
% title={Normalized Torque Setting},
axis x line*=bottom,
axis y line*=left
]
\input{figures/serial_td2.tex}
\end{axis}

\begin{axis}[%
width=5cm,
height=4cm,
at={(7cm,6cm)},
scale only axis,
xmin=0,
xmax=320,
xlabel={Time [s]},
xmajorgrids,
ymin=0,
ymax=0.06,
ylabel={Normalized Recycle Valve Opening},
ylabel style={yshift=-0.1cm},
ymajorgrids,
axis background/.style={fill=white},
% title style={font=\bfseries},
% title={Normalized Recycle Valve Opening},
axis x line*=bottom,
axis y line*=left
]
\input{figures/serial_ur2.tex}
\end{axis}
\end{tikzpicture}

  }
    \caption[Time response of serial system.]{Comparison of serial system time responses of centralized and distributed controllers using 3 solver iterations. Output disturbance is applied at \u{50}{s}.}
  \label{fig:res:serial-timeresp}
\end{figure}

\begin{figure}
  \resizebox{\linewidth}{!}{%
    % This file was created by matlab2tikz.
%
\definecolor{mycolor1}{rgb}{0.00000,0.44700,0.74100}%
\definecolor{mycolor2}{rgb}{0.85000,0.32500,0.09800}%
\definecolor{mycolor3}{rgb}{0.92900,0.69400,0.12500}%
%
\begin{tikzpicture}

\begin{axis}[%
width=5cm,
height=4cm,
at={(0\linewidth,0\linewidth)},
scale only axis,
xmin=50,
xmax=80,
xlabel={Time [s]},
xmajorgrids,
ymin=-6,
ymax=1,
ylabel={Relative Surge Control Distance [\%]},
ymajorgrids,
axis background/.style={fill=white},
title style={font=\bfseries},
title={Upstream Compressor},
axis x line*=bottom,
axis y line*=left
]
% Add downstream compressor data
\input{figures/serial_sd2_zoom_data.tex}
\end{axis}
%
%
%
\begin{axis}[%
width=5cm,
height=4cm,
at={(6cm,0\linewidth)},
scale only axis,
xmin=50,
xmax=80,
xlabel={Time [s]},
xmajorgrids,
ymin=-6,
ymax=1,
ymajorgrids,
axis background/.style={fill=white},
title style={font=\bfseries},
title={Downstream Compressor},
axis x line*=bottom,
axis y line*=left
]
% Add upstream compressor data
\input{figures/serial_sd1_zoom_data.tex}
\end{axis}
%
%
\end{tikzpicture}%

  }
  \caption[Zoomed view of surge distance time response of serial system.]{Zoomed view of serial system surge distance time response given in \fig{res:serial-timeresp}.}
  \label{fig:res:serial-sd-zoom}
\end{figure}

Differences in the controller response and performance can be observed for each of the control approaches.
The cooperative controller response is qualitatively similar to that of the centralized controller.
It is, however, somewhat less aggressive in its regulation of the output pressure, evidenced by its less extreme change from high to low torque in the downstream compressor and faster return to the steady-state torque value, as shown in \fig{res:serial-timeresp}.
As a result, the output pressure of the downstream compressor has a much lower overshoot than either the centralized or non-cooperative controller.
The surge distances of the centralized and cooperative controllers, however, show an almost identical response for the downstream compressor and a similar response upstream, though that of the cooperative controller has a greater increase near \u{58}{s} (see \fig{res:serial-sd-zoom}).

The non-cooperative controller response has a much different characteristic than that of the centralized controller.
There is no significant initial increase in the torque input to either compressor when the disturbance is applied; as a result, both compressors are pushed further towards surge than in the centralized case, and the surge distances are also slower to converge.
The maximum disturbance to the output pressures of both compressors for the non-cooperative controller is accordingly reduced by approximately 40\% when compared to the centralized case.
The reason for the difference in performance is the much higher degree of coupling between the serial compressors as compared to the parallel case.

The integral squared error (ISE) and integral absolute error (IAE) are shown in \tab{res:performance:ser-ise} for all controllers.

\begin{table}
  \centering
  \caption{Integral squared error (ISE) and integral absolute error (IAE) measures for serial controllers.}
  \begin{tabular}{ccccccc}
    \toprule
    & \multicolumn{2}{c}{Centralized} & \multicolumn{2}{c}{Cooperative} & \multicolumn{2}{c}{Non-cooperative}\\
    & ISE & IAE & ISE & IAE &ISE & IAE \\
    \midrule
    \gi{torque} &   0.0012 &    0.012 &    0.001 &   0.0079 &   0.0015 &    0.019\\
    \gi{ur} &  8.6e-05 &   0.0037 &  5.6e-05 &   0.0024 &  0.00011 &   0.0053\\
    \gi{pd} &  0.00096 &    0.011 &   0.0006 &   0.0067 &  0.00081 &    0.014\\
    \gi{sd} &    0.052 &    0.063 &    0.031 &    0.041 &    0.078 &     0.13\\
    \gii{torque} &   0.0027 &    0.016 &  0.00095 &   0.0073 &   0.0017 &    0.021\\
    \gii{ur} &  1.1e-05 &   0.0011 &  2.7e-05 &   0.0016 &  3.2e-05 &   0.0026\\
    \gii{pd} &  0.00047 &   0.0051 &  0.00058 &   0.0049 &  0.00029 &    0.005\\
    \gii{sd} &    0.056 &    0.032 &    0.064 &    0.035 &      0.1 &    0.064\\
    \bottomrule
  \end{tabular}
  \label{tab:res:performance:ser-ise}
\end{table}



\subsection{Computational Efficiency}

The computational cost of the centralized and distributed control approaches was evaluated and is presented in \fig{results:compcost}.%
\footnote{Tests run on a Intel\textregistered{} Core\texttrademark{} i5-540M \u{2.53}{\giga\hertz} processor.} 
The computation times for the distributed controllers assume a parallelized implementation where each sub-controller is executed using a separate processor. 

\begin{figure}
  \resizebox{\linewidth}{!}{%
    % This file was created by matlab2tikz.
%
\definecolor{mycolor1}{rgb}{0.00000,0.44700,0.74100}%
\definecolor{mycolor2}{rgb}{0.85000,0.32500,0.09800}%
\definecolor{mycolor3}{rgb}{0.92900,0.69400,0.12500}%
%
\begin{tikzpicture}

\begin{axis}[%
width=0.38\linewidth,
height=0.3\linewidth,
at={(0\linewidth,0\linewidth)},
scale only axis,
xmin=1,
xmax=9,
xlabel={Number of solver iterations},
xmajorgrids,
ymin=0.2,
ymax=0.55,
ylabel={Computation time [ms]},
ymajorgrids,
axis background/.style={fill=white},
title style={font=\bfseries,yshift=2.1ex},
title={Parallel System},
ylabel near ticks,
legend columns=-1,
legend style={at={(0.03,1.25)},anchor=north west,legend cell align=left,align=left,draw=none}
]
\addplot [color=mycolor1,solid,line width=1pt,mark size=3.0pt,mark=o,mark options={solid}]
  table[row sep=crcr]{%
1	0.3476\\
2	0.3476\\
3	0.3476\\
4	0.3476\\
5	0.3476\\
6	0.3476\\
7	0.3476\\
8	0.3476\\
9	0.3476\\
};
\addlegendentry{Centralized};

\addplot [color=mycolor2,solid,line width=1pt,mark size=2.5pt,mark=square,mark options={solid}]
  table[row sep=crcr]{%
1 0.3636\\
2 0.3760\\
3 0.3845\\
4 0.4085\\
5 0.4130\\
6 0.4191\\
7 0.4384\\
8 0.4381\\
9 0.4502\\
};
\addlegendentry{Cooperative};

\addplot [color=mycolor3,solid,line width=1pt,mark size=3.0pt,mark=x,mark options={solid}]
  table[row sep=crcr]{%
1 0.2302\\
2 0.2399\\
3 0.2496\\
4 0.2622\\
5 0.2691\\
6 0.2806\\
7 0.2900\\
8 0.3055\\
9 0.3137\\
};
\addlegendentry{Non-cooperative};

\end{axis}

\begin{axis}[%
% width=0.761\linewidth,
% height=0.597\linewidth,
width=0.38\linewidth,
height=0.3\linewidth,
at={(0.5\linewidth,0\linewidth)},
scale only axis,
xmin=1,
xmax=9,
xlabel={Number of solver iterations},
xmajorgrids,
ymin=0.2,
ymax=0.55,
% ylabel={Average computation time [ms]},
ymajorgrids,
axis background/.style={fill=white},
title style={font=\bfseries,yshift=2.1ex},
title={Serial System}
]
\addplot [color=mycolor1,solid,line width=1pt,mark size=3.0pt,mark=o,mark options={solid},forget plot]
  table[row sep=crcr]{%
1	0.3713\\
2	0.3713\\
3	0.3713\\
4	0.3713\\
5	0.3713\\
6	0.3713\\
7	0.3713\\
8	0.3713\\
9	0.3713\\
};
\addplot [color=mycolor2,solid,line width=1pt,mark size=2.5pt,mark=square,mark options={solid},forget plot]
  table[row sep=crcr]{%
1 0.3437\\
2 0.3620\\
3 0.3670\\
4 0.3794\\
5 0.3895\\
6 0.4135\\
7 0.4199\\
8 0.4249\\
9 0.4428\\
};
\addplot [color=mycolor3,solid,line width=1pt,mark size=3.0pt,mark=x,mark options={solid},forget plot]
  table[row sep=crcr]{%
1 0.2169\\
2 0.2224\\
3 0.2294\\
4 0.2415\\
5 0.2483\\
6 0.2576\\
7 0.2694\\
8 0.2773\\
9 0.2887\\
};
\end{axis}

\end{tikzpicture}%

  }
  \caption[Controller computation time per iteration.]{Average controller computation time per iteration, as a function of the number of solver iterations used. Operating point is 3 solver iterations. Output disturbance is applied at \u{50}{s} and total simulation time is \u{500}{s}.}
  \label{fig:results:compcost}
\end{figure}


As expected, the non-cooperative controller achieves the lowest computation times for both the parallel and the serial system.
In both cases, the cooperative controller has a computation time approximately on par with the centralized controller.

The significant advantage in computation time demonstrated by the non-cooperative controller is a result of the reduced size of its prediction matrices, which are multiplied to generate the \g{qp}.
The computational cost of the \g{qp} generation is approximately linear in the number of outputs used, and the non-cooperative approach considers fewer outputs than the centralized or cooperative approaches.

As described in~\cite{Jones2016}, for the systems considered here, the QP-generation step has a much higher computational cost than the QP-solving step.
There is thus limited scope for decreasing the required computation time without decreasing the number of states used to generate the QP problem.

