\label{sec:modelling}
The models used in this work are based on a two-compressor system at an ABB Research facility.
The physical setup is described in \cite{Cortinovis2014} and \cite{Cortinovis2015} for a single compressor.
For simplicity, the two compressors in each system were assumed to be identical.

\begin{figure}
  % \begin{subfigure}{\linewidth}
    \togglefalse{compleg}
\togglefalse{complabel}
\resizebox{\linewidth}{!}{%
  \large
  \begin{tikzpicture}
      % draw first compressor
    \renewcommand{\glsgraphcmd}[1]{\glsentryuseri{#1}}
    \drawcomp

      % Draw second compressor shifted
    \begin{scope}[yscale=1,xscale=-1,shift={(-24,0)},nodes={xscale=1}]
      \renewcommand{\glsgraphcmd}[1]{\glsentryuserii{#1}}
      \drawcomp
    \end{scope}

    \renewcommand{\glsgraphcmd}[1]{\glsentryname{#1}}

      % Increase size of bounding box
    % \useasboundingbox ($(current bounding box.south east) - (0,2)$) rectangle (current bounding box.north west |- 0,17);

    % \node[align=center] (origin) at (12.5,3.0) {Common\\Tank\\\statetext{\glsgraph{pt}}};
    \coordinate (origin) at (12.0, 3.4);

    % Draw common tank and valve
    % \drawtank{(ctank-bl)}{(ctank-tr)}
    \node[align=center,rectangle,minimum height=2.5cm,text width=5cm,draw=tankcolor,line width=\tankthick] (ctank) at (origin) {Common Tank\\\statetext{\glsgraph{pt}}};

    \coordinate (ctank-bl) at (ctank.south west);
    \coordinate (ctank-tr) at (ctank.north east);
    \coordinate (ovalve) at ($(ctank.north) + (0,\valvewidth) + (0,1)$);

    \drawvalve[rotate=90, transform shape]{ovalve}{0.4}
    \node[below, rotate=90, transform shape, align=center] at ($(ovalve)+(\valveheight,0)$) {Common\\Tank Valve};

    % Draw flow arrows
    \draw[->,line width=\arrowthick] (ctank.north) -- ($(ovalve)-(0,\valvewidth)-(0,0.5*\valvethick)$);
    \drawflow[->]{($(ovalve)+(0,\valvewidth)+(0,0.5*\valvethick)$)}{+(0,1)}

    % Label valve
    \draw[<-,draw=inputcolor,line width=2pt] ($(ovalve)-(2*\valveheight,0)$) -- ++(-1,0) node[below right]{\inputtext{\glsgraph{ut}}};


    % Label boundary conditions
    \node at ($(ovalve)+(0,2)$) {\bctext{\glsgraph{pa}}};
    \node at ($(ovalve)+(10.5,0)$) {\bctext{\glsgraph{pa}}};
    \node at ($(ovalve)+(-10.5,0)$) {\bctext{\glsgraph{pa}}};

    % Label individual compressors
    \node[align=center] at ($(origin)-(6,1)$) {\Large\bfseries Compressor 1};
    \node[align=center] at ($(origin)+(6,-1)$) {\Large\bfseries Compressor 2};

    % Legend
    % \drawlegend{($(ovalve)+(3.5,0)$)}
    % \drawlegend{(current bounding box.north east)}

  \end{tikzpicture}
}

    {\centering a) Parallel compressor system.\\}
  % \end{subfigure}
  % \\
  % \begin{subfigure}{\linewidth}
    \togglefalse{compleg}
\togglefalse{complabel}
\newdimen\xcoord
\newdimen\ycoord
\newdimen\xcoordb
\newdimen\ycoordb

  \centering
\resizebox{0.9\linewidth}{!}{%
  \begin{tikzpicture}
    \renewcommand{\glsgraphcmd}[1]{\glsentryuseri{#1}}
      % draw first compressor
    \drawcomp
      % \draw (current bounding box.south west) rectangle (current bounding box.north east);

    \togglefalse{compfirstvalve}

    \pgfgetlastxy{\xcoord}{\ycoord}
    \renewcommand{\glsgraphcmd}[1]{\glsentryuserii{#1}}

      % Draw second compressor shifted
    \begin{scope}[yscale=-1,xscale=1,shift={($(\xcoord,\ycoord) - (0.5*\tankdim,6.65)$)},transform shape,nodes={xscale=1,yscale=-1}]
      \drawcomp
    \end{scope}

    \pgfgetlastxy{\xcoordb}{\ycoordb}

    % \coordinate (origin) at ($0.5*(\xcoord,\ycoord) + 0.5*(\xcoordb,\ycoordb)$);
    \coordinate (origin) at (\xcoord,\ycoord);
    % Label compressors
    \node at ($(origin) - (4,0)$) {\Large \bfseries Compressor 1};
    \node at ($(origin) + (4,0)$) {\Large \bfseries Compressor 2};
    \node at ($(origin) + (6.2,-0.2)$) {\bctext{\glsgraph{pa}}};
    \node at ($(origin) - (6.0,-2.7)$) {\bctext{\glsgraph{pa}}};

    % Legend
    % \drawlegend{($(current bounding box.north west) + (1,-1)$)}

  \end{tikzpicture}
}
\\
    {\centering b) Serial compressor system.\\}
  % \end{subfigure}
  \caption[Diagram of compressor systems.]{Diagram of compressor systems with inputs and states labelled.}
  \label{fig:comp-systems-diagrams}
\end{figure}


The compressor model used in this work is a centrifugal compressor driven by a variable-speed electric driver, with a tank at both the inlet (suction) and outlet (discharge).
Both the suction and discharge tanks are connected to atmospheric pressure by a valve.
These suction and discharge valves represent the flow conditions upstream and downstream of the compressor, such that various flow conditions and disturbances can be simulated by changing the position of the valves.
The artificial disturbances applied using these valves are designed to mimic situations frequently encountered in industrial applications (e.g.\ shutdown or startup of a downstream compressor).
The compressor model used is based on the Gravdahl-Greitzer model described in \cite{Gravdahl1999}. 
It is defined by five states: the suction (\g{ps}) and discharge (\g{pd}) pressures, the mass flow through the compressor (\g{qc}), the rotational speed of the compressor (\g{omegac}), and the mass flow rate through the recycle valve (\g{qr}).
The two inputs to the system are the torque applied to the electric driver (\g{torque}) and the position of the recycle valve (\g{ur}).
The controlled outputs are the output pressure (\g{pout}) and the surge distance (\g{sd}), a measure of how far the compressor is from surge.
The interested reader may refer to \cite{Cortinovis2015} for a more detailed description of the model and its identification.

The parallel and serial compressor systems studied in this work are shown in Figure~\ref{fig:comp-systems-diagrams}.
In the parallel case, the two compressors' outlet tanks both discharge into a single discharge tank and the pressure in this discharge tank (\g{pt}) is the primary process variable.
This pressure also serves as an additional state while the inputs and outputs remain the same as for a single compressor.
For the serial configuration, the upstream compressor discharges directly into the inlet of the downstream compressor.
The states, inputs and outputs are the same as those of the individual compressors.
