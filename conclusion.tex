\label{sec:conclusion}

In this work, the development of a distributed model predictive control scheme combining process and anti-surge control for systems of compressors was presented.
Two compressor networks were studied: a parallel and a serial configuration, each with two compressors.
The control scheme was based on linearized MPC using a linear model of the system updated at each time step.
Distributed controllers using both cooperative and non-cooperative cost functions were developed.

The controllers were implemented in \slink{} and in \cpp{} using the QP solver \qpoases{}.
Both implementations were developed to be suitable for deployment on embedded hardware.
In the \cpp{} implementation in particular, the known structure of the compressor model matrices was taken into account to reduce computation time.

The performance of the distributed controllers was then evaluated in simulation and compared to the benchmark established by a centralized MPC controller.
Both the time responses of the distributed controllers as well as their computation time were evaluated.
For the parallel system, both cooperative and non-cooperative controllers achieved virtually identical performance to the centralized controller.
The cooperative controller had a 11\% higher average computation time than in the centralized case, while the non-cooperative scheme reduced the average computation time by 28\%.

The serial system had different controller performances for each of the centralized, cooperative and non-cooperative controllers.
The cooperative controller achieved very similar performance in the time response as the centralized controller, with $<1\%$ difference in both the maximum discharge pressure and minimum surge control distance in the downstream compressor.
The non-cooperative controller also had a qualitatively similar response, however it did not perform as well in anti-surge control, allowing a 9\% decrease in the minimum surge control distance reached in the downstream compressor compared to the centralized controller.
Its reduced performance in surge control was combined with improved process control: its maximum discharge pressure was 40\% lower than in the centralized case.
Again, the non-cooperative controller outperformed the cooperative controller in terms of computation time, giving a 38\% decrease compared to the centralized case, while the cooperative controller only achieved a 1\% reduction.  

The computational efficiency of the distributed controllers could be further improved by investigating the effect of reducing the model order at the sub-controller level on overall controller performance.
In particular, reducing the number of states used to generate the prediction matrices could lead to relatively high performance gains as the prediction matrix generation is $\mathcal{O}(n^3)$ in the number of states.
% On the implementation side, the computational cost could be reduced by replacing the \eigen{} linear algebra library with a \texttt{BLAS}-based library, which is the benchmark for computational efficiency when performing basic linear algebra computations.

