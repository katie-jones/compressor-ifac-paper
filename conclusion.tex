\label{sec:conclusion}

In this work, the development of a distributed model predictive control scheme combining process and anti-surge control for systems of compressors was presented.
Two compressor networks were studied: a parallel and a serial configuration, each with two compressors.
% The control scheme was based on linearized MPC using a linear model of the system updated at each time step.
Distributed controllers using both cooperative and non-cooperative cost functions were developed.

The performance of the distributed controllers was then evaluated in simulation and compared to the benchmark established by a centralized MPC controller.
For the parallel system, both dMPC controllers achieved virtually identical control performance compared to the centralized controller.
The cooperative controller had a 11\% higher average computation time than in the centralized case, while the non-cooperative controller reduced it by 28\%.

The serial system had different controller performances for each of the three controllers.
The cooperative controller achieved very similar performance in the time response as the centralized controller, with less than $1\%$ difference in both the maximum discharge pressure and minimum surge control distance in the downstream compressor.
The non-cooperative controller also had a qualitatively similar response, however it did not perform as well in anti-surge control, with a 9\% decrease in the minimum surge control distance reached in the downstream compressor compared to the centralized controller.
Its reduced performance in surge control was combined with improved process control: its maximum discharge pressure was 40\% lower than in the centralized case.
Again, the non-cooperative controller outperformed the cooperative controller in terms of computation time, giving a 38\% decrease compared to the centralized case, while the cooperative controller only achieved a 1\% reduction.  

The computational efficiency of the distributed controllers could be further improved by investigating the effect of reducing the model order at the sub-controller level on controller performance.
In particular, reducing the number of states used to generate the prediction matrices could lead to significant performance gains as the prediction matrix generation is $\mathcal{O}(n^3)$ in the number of states.
